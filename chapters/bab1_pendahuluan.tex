% chtex-lint: disable
\chapter{PENDAHULUAN}

\section{Latar Belakang}

Latar belakang tesis magister pada dasarnya adalah “cerita ilmiah” yang menjelaskan mengapa topik yang Anda teliti penting, apa yang bermasalah, dan di mana posisi penelitian Anda di antara karya-karya sebelumnya. Penulisannya mengikuti alur deduktif: mulai dari fenomena umum, mengerucut ke konteks yang lebih spesifik, lalu berakhir pada perumusan masalah dan kebutuhan penelitian.

Secara umum, struktur naratif latar belakang dapat diatur sebagai berikut:

Paragraf awal berisi gambaran besar fenomena atau tren di bidang yang Anda teliti (misalnya perkembangan AI, computer vision, sistem pendidikan, dan sebagainya), ditulis dengan kalimat pembuka yang kuat dan informatif.

Paragraf-paragraf berikutnya membawa pembaca ke konteks yang lebih sempit: kondisi nasional, institusional, atau domain tertentu yang menjadi fokus tesis, sekaligus menunjukkan adanya masalah nyata (misalnya kinerja sistem belum optimal, kualitas layanan masih rendah, atau dampak negatif yang belum tertangani). Di sini, Anda menjelaskan urgensi penelitian, yaitu mengapa masalah ini penting untuk segera dikaji, baik dari sisi teoretis maupun praktis.

Setelah itu, latar belakang perlu menonjolkan gap penelitian. Gap ini bisa berupa: topik yang belum banyak dikaji, konteks lokal yang belum disentuh, metode yang belum pernah diterapkan, atau kelemahan studi-studi terdahulu (misalnya data terbatas, metode kurang akurat, variabel penting belum dipertimbangkan). Paparkan gap secara eksplisit tetapi singkat, sehingga terlihat jelas bahwa masih ada ruang yang perlu diisi oleh penelitian Anda. Dari penjelasan urgensi dan gap inilah nantinya rumusan masalah dan tujuan penelitian menjadi terasa “wajar” dan logis.

Dari sisi aturan penulisan, gunakan bahasa ilmiah yang formal, objektif, dan konsisten (hindari “saya” atau opini pribadi yang tidak berdasar). Setiap klaim faktual idealnya didukung rujukan, tetapi detail teoritis dan teknis yang sangat spesifik sebaiknya disimpan untuk bab tinjauan pustaka. Latar belakang dibuat mengalir antar paragraf dengan penghubung yang jelas, tidak berupa poin-poin, dan biasanya memiliki panjang yang cukup untuk memberikan konteks yang matang (seringnya sekitar beberapa halaman dalam proporsi yang wajar terhadap keseluruhan bab pendahuluan). Dengan cara ini, latar belakang menjadi fondasi yang kokoh untuk memahami rumusan masalah, tujuan, dan keseluruhan rancangan penelitian.

\section{Rumusan Masalah}

Rumusan masalah dalam tesis magister menekankan penyajian berupa narasi deskriptif yang afirmatif dan terstruktur, bukan daftar pertanyaan, untuk menggambarkan secara jelas kesenjangan penelitian yang muncul dari latar belakang, sehingga menjadi jembatan logis menuju tujuan penelitian. Mulailah dengan mengintegrasikan elemen-elemen masalah secara koheren dalam paragraf utuh, yang mencakup identifikasi fenomena utama, variabel kunci, keterbatasan studi sebelumnya, serta implikasi dampaknya jika tidak ditangani, dengan panjang ideal 1-2 paragraf (sekitar 300-500 kata) yang mengalir dari umum ke spesifik tanpa menggunakan kata tanya. Pastikan narasi ini spesifik, terukur, dan dapat diuji melalui metode yang dipilih, sambil menunjukkan novelty penelitian seperti aplikasi konteks lokal, serta hindari pengulangan latar belakang dengan fokus pada fokus operasional masalah.

\textbf{Contoh Narasi Rumusan Masalah} \\
Model deep learning untuk deteksi objek pada citra dengan pencahayaan rendah masih menghadapi tantangan signifikan karena penurunan akurasi yang drastis, sehingga menyebabkan ketidakandalan pada aplikasi seperti pengawasan keamanan dan kendaraan otonom. Penurunan performa ini terutama disebabkan oleh kurangnya data pelatihan yang representatif dan ketidakmampuan algoritma saat menghadapi noise dan variasi intensitas cahaya. Selain itu, metode augmentasi data yang ada belum cukup efektif untuk memperbaiki kekurangan tersebut, sehingga diperlukan pengembangan teknik baru yang lebih robust dan adaptif terhadap kondisi pencahayaan ekstrem. Oleh karena itu, penelitian ini akan fokus pada pengembangan model Computer Vision yang dapat meningkatkan akurasi deteksi objek pada citra pencahayaan rendah dengan memanfaatkan teknik augmentasi data yang inovatif dan algoritma pembelajaran yang lebih adaptif.

Narasi seperti ini memperjelas fokus penelitian secara spesifik dan kontekstual di bidang Computer Vision, menciptakan landasan kuat untuk arah metodologi dan analisis yang akan dilakukan. Pemilihan bahasa harus formal dan objektif, serta menjelaskan urgensi penelitian tanpa menggunakan format tanya, untuk memperkuat argumen ilmiah tesis.

\section{Tujuan Penelitian}

Tujuan penelitian disusun untuk menjelaskan secara jelas apa yang ingin dicapai oleh tesis, sebagai turunan langsung dari rumusan masalah dan selaras dengan latar belakang. Penulisannya menggunakan kalimat pernyataan (bukan pertanyaan), bersifat spesifik, terukur, dan realistis dicapai dengan metode yang digunakan. Biasanya diawali dengan satu kalimat pembuka singkat, kemudian diikuti poin-poin tujuan dalam bentuk daftar agar mudah dibaca.

Secara umum, tata caranya bisa mengikuti pola berikut:
Pastikan setiap tujuan terkait langsung dengan rumusan masalah (satu rumusan masalah minimal satu tujuan). Bedakan bila perlu antara tujuan umum dan tujuan khusus (tujuan umum 1 kalimat, tujuan khusus dalam beberapa poin).
Gunakan kata kerja operasional yang jelas, seperti: menganalisis, menguji, mengukur, membandingkan, mengembangkan, merancang, mengevaluasi. Hindari tujuan yang terlalu luas atau normatif (misalnya “untuk memperbaiki pendidikan Indonesia”), dan fokus pada hal yang benar-benar dihasilkan tesis.

\section{Tujuan Penelitian}

Tujuan penelitian ini disusun sebagai turunan langsung dari rumusan masalah dan berfokus pada pengembangan serta evaluasi penerapan kecerdasan buatan dalam domain yang diteliti. Secara umum, tujuan penelitian dalam bidang kecerdasan buatan dapat dirumuskan sebagai berikut.

\begin{enumerate}
  \item Menganalisis kebutuhan sistem kecerdasan buatan pada domain tertentu (misalnya pendidikan, kesehatan, atau industri) sebagai dasar perancangan solusi berbasis AI.
  \item Merancang dan mengimplementasikan model kecerdasan buatan (seperti \textit{machine learning}, \textit{deep learning}, atau \textit{reinforcement learning}) yang sesuai dengan karakteristik data dan permasalahan yang dikaji.
  \item Mengevaluasi kinerja model kecerdasan buatan yang dikembangkan menggunakan metrik yang relevan (misalnya akurasi, presisi, recall, F1-score, atau metrik lain yang sesuai).
  \item Membandingkan hasil kinerja model yang diusulkan dengan metode atau pendekatan yang sudah ada untuk mengetahui keunggulan dan keterbatasan solusi yang dikembangkan.
  \item Memberikan rekomendasi pengembangan lebih lanjut dan potensi penerapan model kecerdasan buatan dalam konteks nyata sesuai domain aplikasi yang diteliti.
\end{enumerate}

Bagian Manfaat Penelitian menjelaskan “untungnya apa dan untuk siapa” dari hasil tesis yang Anda kerjakan. Isinya menjabarkan kontribusi yang diharapkan, baik untuk pengembangan ilmu (teoretis) maupun untuk pemecahan masalah nyata (praktis). Biasanya diawali satu paragraf pengantar singkat, lalu diikuti butir-butir manfaat agar mudah terbaca.

Secara umum, penataannya dapat dibuat seperti ini:

Jelaskan terlebih dahulu bahwa manfaat penelitian dibagi menjadi dua: manfaat teoretis dan manfaat praktis.

\begin{enumerate}
    \item \textbf{Manfaat Teoretis}: Menjelaskan kontribusi penelitian terhadap pengembangan ilmu, model, metode, atau kerangka konseptual di bidang kecerdasan buatan, seperti memberikan pemahaman baru tentang efektivitas algoritma tertentu atau mengembangkan kerangka kerja inovatif untuk analisis data.
    \item \textbf{Manfaat Praktis}: Merinci dampak konkret dan pihak-pihak yang diuntungkan secara langsung dari hasil penelitian, misalnya membantu guru dalam mengoptimalkan proses pembelajaran, mendukung pengembang sistem dalam menciptakan aplikasi yang lebih efisien, atau memberikan rekomendasi bagi institusi dan perusahaan dalam pengambilan keputusan berbasis data.
\end{enumerate}

Hindari klaim berlebihan; tulis manfaat yang realistis dan benar-benar terkait hasil yang mungkin dicapai oleh penelitian.
\section{Batasan Penelitian}

Agar penelitian ini terfokus dan dapat dilaksanakan secara realistis, maka ruang lingkup penelitian dibatasi oleh beberapa hal sebagai berikut.

\begin{enumerate}
  \item Penelitian ini hanya membahas permasalahan sesuai dengan rumusan masalah yang telah ditetapkan, sehingga tidak mencakup isu-isu lain di luar fokus kajian.
  \item Data yang digunakan dalam penelitian dibatasi pada sumber dan periode tertentu yang telah ditentukan sebelumnya, sehingga hasil yang diperoleh hanya berlaku dalam konteks tersebut.
  \item Variabel yang dianalisis dibatasi pada variabel-variabel yang telah didefinisikan secara operasional, sementara variabel lain yang berpotensi berpengaruh tidak dibahas secara mendalam.
  \item Metode yang digunakan dalam penelitian dibatasi pada pendekatan yang telah dipilih oleh peneliti, sehingga tidak melakukan perbandingan dengan seluruh metode atau teknik analisis yang tersedia.
  \item Hasil penelitian ini dibatasi pada kondisi dan asumsi yang digunakan selama proses pengumpulan dan pengolahan data, sehingga generalisasi temuan perlu dilakukan dengan mempertimbangkan keterbatasan tersebut.
\end{enumerate}
