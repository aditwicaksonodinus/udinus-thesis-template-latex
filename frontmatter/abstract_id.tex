% chtex-lint: disable
% ============================================================================
%                        ABSTRAK (BAHASA INDONESIA)
% ============================================================================
% Halaman abstrak dalam Bahasa Indonesia
%
% PETUNJUK:
% - Maksimum 200 kata
% - Isi harus mencakup:
%   (1) Masalah utama dan ruang lingkup penelitian
%   (2) Metode yang digunakan
%   (3) Hasil yang diperoleh
%   (4) Kesimpulan utama dan saran
% - Kata kunci dipisahkan dengan titik koma (;)
%
% VARIABEL YANG DIGUNAKAN:
% - \ThesisTitle    : Judul tesis (dari main.tex)
% - \AuthorName     : Nama penulis (dari main.tex)
% ============================================================================

\chapter*{ABSTRAK}
\addcontentsline{toc}{chapter}{ABSTRAK}

% Header abstrak dengan judul dan nama penulis
\noindent\textbf{Judul:} \ThesisTitle\\
\textbf{Penulis:} \AuthorName\\[0.75\baselineskip]

% ============================================================================
% ISI ABSTRAK - GANTI TEKS DI BAWAH INI DENGAN ABSTRAK ANDA
% ============================================================================
% PETUNJUK: Hapus teks placeholder di bawah dan ganti dengan abstrak tesis Anda
% Maksimum 200 kata, mencakup: masalah, metode, hasil, dan kesimpulan
% ============================================================================

{Isi abstrak dalam Bahasa Indonesia di sini. Abstrak (maksimum 200 kata) memuat: (1) masalah utama dan ruang lingkup penelitian, (2) metode yang digunakan, (3) hasil yang diperoleh, dan (4) kesimpulan utama serta saran.}

% ============================================================================
% KATA KUNCI - GANTI DENGAN KATA KUNCI TESIS ANDA
% ============================================================================
% PETUNJUK: Tuliskan 3-5 kata kunci yang relevan, dipisahkan dengan titik koma
% ============================================================================

\textbf{Kata kunci:} {kata kunci 1}; {kata kunci 2}; {kata kunci 3}; {kata kunci 4}; {kata kunci 5}.
