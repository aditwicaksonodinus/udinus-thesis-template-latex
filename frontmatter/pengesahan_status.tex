% chtex-lint: disable
% ============================================================================
%                      PENGESAHAN STATUS TESIS
% ============================================================================
% Halaman pengesahan status kerahasiaan tesis
%
% PETUNJUK:
% - Halaman ini berisi pernyataan status kerahasiaan tesis
% - Pilih salah satu status: Sangat Rahasia, Rahasia, atau Biasa
% - Untuk menandai pilihan, ganti \fbox{\phantom{X}} dengan \fbox{$\checkmark$}
%   atau \fbox{\textbf{X}}
%
% VARIABEL YANG DIGUNAKAN:
% - \ThesisTitle        : Judul tesis (dari main.tex)
% - \AuthorName         : Nama penulis huruf kapital (dari main.tex)
% - \AuthorAddress      : Alamat tetap penulis (dari main.tex)
% - \SupervisorOne      : Nama pembimbing utama (dari main.tex)
% - \SupervisorOneNIDN  : NIDN pembimbing utama (dari main.tex)
% ============================================================================

\cleardoublepage
\phantomsection
\addcontentsline{toc}{chapter}{PENGESAHAN STATUS TESIS}
\thispagestyle{plain}

\begin{center}
% Logo UDINUS
\includegraphics[width=3cm]{documents/udinus-logo.png}

\vspace{0.2em}

% Nama Universitas
{\bfseries UNIVERSITAS DIAN NUSWANTORO}

\vspace{0.5em}

% Judul Halaman
{\bfseries PENGESAHAN STATUS TESIS}

\end{center}

\vspace{0.2cm}

% ============================================================================
% PERNYATAAN STATUS
% ============================================================================

Tesis dengan Judul \textbf{\ThesisTitle}, Saya \AuthorName. Mengijinkan Tesis Magister Komputer ini disimpan di Perpustakaan Universitas Dian Nuswantoro dengan syarat-syarat ketentuan sebagai berikut:

\begin{enumerate}
\item Tesis adalah hak milik Universitas Dian Nuswantoro
\item Perpustakaan Universitas Dian Nuswantoro dibenarkan membuat salinan untuk tujuan referensi saja.
\item Perpustakaan juga dibenarkan membuat salinan Tesis ini sebagai bahan pertukaran antarinstitusi pendidikan tinggi.
\end{enumerate}

\vspace{0.1em}

% ============================================================================
% PILIHAN STATUS KERAHASIAAN
% ============================================================================
% PETUNJUK: Tandai salah satu pilihan dengan mengganti \fbox{\phantom{X}}
%           menjadi \fbox{$\checkmark$} atau \fbox{\textbf{X}}
% ============================================================================

% Di bagian isi dokumen:
\noindent\begin{tabularx}{\textwidth}{@{\hspace{0.5cm}}l@{\hspace{0.5cm}}X}
\fbox{\phantom{X}} & \textbf{Sangat Rahasia} (Mengandung isi tentang keselamatan atau kepentingan Negara Republik Indonesia) \\[0.3cm]
\fbox{\phantom{X}} & \textbf{Rahasia} (Mengandung isi tentang kerahasiaan dari suatu organisasi/badan tempat penelitian Tesis ini dikerjakan) \\[0.3cm]
\fbox{\textbf{X}} & \textbf{Biasa}
\end{tabularx}

\vspace{1cm}

\noindent \hspace{7cm} Disahkan oleh:

\vspace{2cm}

% ============================================================================
% BAGIAN TANDA TANGAN
% ============================================================================

\noindent
\begin{tabular*}{\textwidth}{@{}p{4.5cm}@{\extracolsep{\fill}}p{9cm}@{}}
\AuthorNameB & \SupervisorOne \\[0.1em]
\cline{2-2}
\AuthorAddress & \textbf{Pembimbing Utama} \\[0.1em]
& NIDN \SupervisorOneNIDN
\end{tabular*}

