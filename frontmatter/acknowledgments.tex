% chtex-lint: disable
% ============================================================================
%                     KATA PENGANTAR (ACKNOWLEDGMENTS)
% ============================================================================
% Halaman kata pengantar/ucapan terima kasih
%
% PETUNJUK:
% - Tuliskan ucapan terima kasih kepada pihak yang membantu
% - Biasanya mencakup:
%   1. Tuhan Yang Maha Esa
%   2. Keluarga
%   3. Dosen pembimbing
%   4. Dosen penguji
%   5. Teman dan kolega
%   6. Pihak lain yang membantu
% - Akhiri dengan harapan tesis bermanfaat
%
% VARIABEL YANG DIGUNAKAN:
% - \ThesisTitle    : Judul tesis (dari main.tex)
% - \AuthorNameB    : Nama penulis huruf kapital di awal (dari main.tex)
% ============================================================================

\chapter*{KATA PENGANTAR}
\addcontentsline{toc}{chapter}{KATA PENGANTAR}

% ============================================================================
% ISI KATA PENGANTAR - SESUAIKAN DENGAN TESIS ANDA
% ============================================================================
% PETUNJUK: Sesuaikan isi kata pengantar dengan kondisi Anda
% ============================================================================

Tesis dengan judul \textbf{\ThesisTitle} ini dapat penulis selesaikan sesuai rencana karena dukungan dari berbagai pihak yang tidak ternilai besarnya. Oleh karena itu penulis menyampaikan terima kasih kepada:

\begin{enumerate}
    % --- Ucapan Terima Kasih Kepada Tuhan ---
    \item {Tuhan Yang Maha Esa yang telah memberikan rahmat dan karunia-Nya sehingga penulis dapat menyelesaikan tesis ini dengan baik.}

    % --- Ucapan Terima Kasih Kepada Keluarga ---
    \item {Keluarga tercinta yang telah memberikan dukungan moral, doa, dan motivasi yang tiada henti selama proses penulisan tesis ini.}

    % --- Ucapan Terima Kasih Kepada Pembimbing Utama ---
    % PETUNJUK: Ganti dengan nama pembimbing utama Anda
    \item {{Nama Pembimbing Utama}, selaku pembimbing utama yang telah memberikan arahan, masukan, dan bimbingan yang sangat berharga dalam penyusunan tesis ini.}

    % --- Ucapan Terima Kasih Kepada Pembimbing Kedua ---
    % PETUNJUK: Ganti dengan nama pembimbing kedua Anda
    \item {{Nama Pembimbing Kedua}, selaku pembimbing kedua yang telah meluangkan waktu dan memberikan perspektif berbeda serta saran konstruktif dalam penyempurnaan tesis ini.}

    % --- Ucapan Terima Kasih Kepada Pihak Lain ---
    % PETUNJUK: Tambahkan pihak lain yang membantu (validator, narasumber, dll)
    \item {Pihak-pihak yang membantu dalam pelaksanaan penelitian, khususnya dalam proses pengumpulan data dan validasi instrumen penelitian.}

    % --- Ucapan Terima Kasih Umum ---
    \item Semua pihak yang tidak dapat disebutkan satu per satu, yang telah memberikan bantuan, dukungan, dan kontribusi dalam penyelesaian tesis ini.
\end{enumerate}

Penulis menyadari adanya keterbatasan penelitian ini, maka kritik, saran, dan masukan yang membangun akan sangat membantu penulis dalam penelitian selanjutnya. Semoga tulisan ini dapat bermanfaat bagi ilmu pengetahuan dan pembaca.

% ============================================================================
% TANDA TANGAN PENULIS
% ============================================================================
% PETUNJUK: Isi tanggal saat tesis selesai
% ============================================================================

\vspace{1cm}
\begin{flushright}
{Kota}, \rule{2.5cm}{0.4pt}\\  % Ganti {Kota} dengan kota Anda
\vspace{3em}
\AuthorNameB
\end{flushright}
