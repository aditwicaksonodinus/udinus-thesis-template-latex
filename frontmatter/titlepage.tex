% chtex-lint: disable
% ============================================================================
%                         HALAMAN JUDUL (TITLE PAGE)
% ============================================================================
% Halaman judul dalam tesis, dicetak pada kertas HVS putih
%
% PETUNJUK:
% - File ini menggunakan variabel dari main.tex
% - Pastikan variabel di main.tex sudah diisi dengan benar
% - Logo UDINUS: documents/udinus-logo.png (pastikan file ada)
%
% VARIABEL YANG DIGUNAKAN:
% - \ThesisTitle    : Judul tesis (dari main.tex)
% - \ThesisPurpose  : Tujuan tesis (dari main.tex)
% - \AuthorName     : Nama penulis huruf kapital (dari main.tex)
% - \StudentID      : NIM penulis (dari main.tex)
% - \City           : Kota (dari main.tex)
% - \Year           : Tahun (dari main.tex)
% ============================================================================

\thispagestyle{empty} % Sembunyikan nomor halaman pada title page

\begin{center}

  \vspace*{2pt}

  % Label "TESIS" - 14pt Bold
  {\fontsize{14pt}{21pt}\selectfont\bfseries TESIS\par}

  \vspace{1cm}

  % Judul Tesis - 16pt Bold, Uppercase
  {\fontsize{16pt}{24pt}\selectfont\bfseries\MakeUppercase{\ThesisTitle}\par}

  \vspace{1.5cm}

  % Tujuan Tesis - Ukuran normal
  {\normalsize\ThesisPurpose\par}

  \vspace{1cm}

  % Nama Penulis dan NIM - Ukuran normal
  {\normalsize\AuthorName\\[0.3em]NIM: \StudentID\par}

  \vspace{1.7cm}

  % Logo UDINUS
  \includegraphics[width=4cm]{documents/udinus-logo.png}\\[1cm]

  \vspace{2cm}

  % Informasi Program Studi - 14pt Bold
  {\fontsize{14pt}{22pt}\selectfont\bfseries
    PROGRAM PASCASARJANA\\[-0.5em]
    MAGISTER TEKNIK INFORMATIKA\\[-0.5em]
    UNIVERSITAS DIAN NUSWANTORO\\[-0.5em]
    \City\\[-0.5em]
    \Year
  \par}

\end{center}

% Pindah ke halaman baru (pengesahan di halaman berikutnya)
\clearpage
