% chtex-lint: disable
% ============================================================================
%                        PENGESAHAN TESIS
% ============================================================================
% Halaman pengesahan tesis setelah sidang
%
% PETUNJUK:
% - Halaman ini ditandatangani setelah sidang tesis
% - Ditandatangani oleh: Dewan Penguji, Dekan, dan Ketua Penguji
% - Isi tanggal sidang pada teks pengesahan dan bagian tanda tangan
%
% VARIABEL YANG DIGUNAKAN:
% - \ThesisTitle    : Judul tesis (dari main.tex)
% - \AuthorName     : Nama penulis huruf kapital (dari main.tex)
% - \StudentID      : NIM penulis (dari main.tex)
% - \ExaminerOne    : Nama penguji 1 (dari main.tex)
% - \ExaminerTwo    : Nama penguji 2 (dari main.tex)
% - \DeanName       : Nama dekan (dari main.tex)
% - \ChairExaminer  : Nama ketua penguji (dari main.tex)
% ============================================================================

\cleardoublepage
\phantomsection
\addcontentsline{toc}{chapter}{PENGESAHAN TESIS}
\thispagestyle{plain}

\begin{center}
% Logo UDINUS
\includegraphics[width=3cm]{documents/udinus-logo.png}

\vspace{0.5cm}

% Nama Universitas
{\bfseries UNIVERSITAS DIAN NUSWANTORO}

\vspace{1cm}

% Judul Halaman
{\bfseries PENGESAHAN TESIS}

\end{center}

\vspace{0.8cm}

% ============================================================================
% METADATA TESIS
% ============================================================================

\noindent
\begin{tabular}{@{}l@{\hspace{0.5cm}}l@{\hspace{0.5cm}}p{\dimexpr\textwidth-1cm-1.5cm\relax}@{}}
JUDUL & : & \textbf{\ThesisTitle} \\[0.3cm]
NAMA & : & \AuthorName \\[0.3cm]
NIM & : & \StudentID
\end{tabular}

\vspace{0.8cm}

% ============================================================================
% TEKS PENGESAHAN
% ============================================================================
% PETUNJUK: Ganti {Tanggal Sidang} dengan tanggal sidang yang sebenarnya
%           Contoh: "25 Maret 2025"
% ============================================================================

\noindent
\hspace{0.5cm} Tesis ini telah diujikan dan dipertahankan di hadapan Dewan Penguji pada Sidang Tesis tanggal {Tanggal Sidang}. Menurut pandangan kami, Tesis ini memadai dari segi kualitas untuk tujuan penganugerahan gelar Magister Komputer (M.Kom.)

\vspace{0.8cm}

% ============================================================================
% BAGIAN TANDA TANGAN DEWAN PENGUJI
% ============================================================================
% PETUNJUK: Isi tanggal pengesahan
% ============================================================================

\begin{center}
{Kota}, \makebox[4cm]{\dotfill}  % Ganti {Kota} dengan kota Anda

\vspace{0.5cm}

Dewan Penguji:
\end{center}

\vspace{1.5cm}

% --- Baris Pertama: 2 Kolom Anggota Penguji ---
\noindent
\begin{minipage}[t]{0.49\textwidth}
\centering
{\small \underline{\ExaminerOne}}\\
{\small Anggota}
\end{minipage}%
\hfill
\begin{minipage}[t]{0.49\textwidth}
\centering
{\small \underline{\ExaminerTwo}}\\
{\small Anggota}
\end{minipage}

\vspace{3em}

% --- Baris Kedua: Mengetahui ---
\begin{center}
{\small Mengetahui,}
\end{center}

\vspace{3em}

% --- Baris Ketiga: 2 Kolom Dekan dan Ketua Penguji ---
\noindent
\begin{minipage}[t]{0.49\textwidth}
\centering
{\small \underline{\DeanName}}\\
{\small Dekan}
\end{minipage}%
\hfill
\begin{minipage}[t]{0.49\textwidth}
\centering
{\small \underline{\ChairExaminer}}\\
{\small Ketua Penguji}
\end{minipage}
