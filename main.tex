% chtex-lint: disable
% !TEX TS-program = xelatex
% ============================================================================
%                    TEMPLATE TESIS UDINUS (MAGISTER TEKNIK INFORMATIKA)
% ============================================================================
% Disusun mengikuti Panduan Tesis UDINUS (2020)
%
% PETUNJUK KOMPILASI:
% 1. XeLaTeX  -> Kompilasi pertama
% 2. Biber    -> Untuk bibliografi (biblatex)
% 3. XeLaTeX  -> Kompilasi kedua
% 4. XeLaTeX  -> Kompilasi ketiga (untuk cross-reference)
%
% CATATAN:
% - Template ini menggunakan XeLaTeX untuk mendukung font Times New Roman
% - Pastikan font Times New Roman terinstal di sistem Anda
% - Gunakan editor LaTeX yang mendukung XeLaTeX (TeXstudio, VS Code + LaTeX Workshop, dll)
% ============================================================================

\documentclass[12pt,oneside]{book}

% ============================================================================
%                           PENGATURAN HALAMAN
% ============================================================================
% Margin sesuai panduan UDINUS:
% - Kiri: 4 cm (untuk jilid)
% - Kanan: 2.5 cm
% - Atas: 3 cm
% - Bawah: 3 cm
\usepackage[a4paper,left=4cm,right=2.5cm,top=3cm,bottom=3cm]{geometry}
\usepackage{tabularx}
% ============================================================================
%                           PENGATURAN FONT
% ============================================================================
% Font wajib: Times New Roman 12pt (sesuai panduan)
\usepackage{fontspec}
\setmainfont{Times New Roman}
\usepackage{unicode-math}
\setmathfont{Latin Modern Math}

% ============================================================================
%                           WARNA KHUSUS COVER
% ============================================================================
\usepackage{xcolor}
\definecolor{darkred}{RGB}{139,0,0}  % Warna merah gelap untuk cover

% ============================================================================
%                           PENGATURAN SPASI
% ============================================================================
% Spasi 1.5 sesuai panduan penulisan tesis
\usepackage{setspace}
\onehalfspacing
\usepackage{indentfirst}  % Indent paragraf pertama setiap section
\usepackage{enumitem}
\setlist[enumerate]{leftmargin=*, itemsep=0.5em}
\setlist[itemize]{leftmargin=*, itemsep=0.5em}

% ============================================================================
%                           PENGATURAN BAHASA
% ============================================================================
\usepackage[english,bahasai]{babel}

% ============================================================================
%                        HEADER & FOOTER (NOMOR HALAMAN)
% ============================================================================
% Nomor halaman: kanan atas (untuk halaman biasa)
%                tengah bawah (untuk awal bab/chapter)
\usepackage{fancyhdr}
\pagestyle{fancy}
\fancyhf{}
\fancyhead[R]{\thepage}  % Nomor halaman di kanan atas
\renewcommand{\headrulewidth}{0pt}
\setlength{\headheight}{14.5pt}

% Gaya halaman awal bab (plain) -> nomor di tengah bawah
\fancypagestyle{plain}{%
  \fancyhf{}
  \fancyfoot[C]{\thepage}
  \renewcommand{\headrulewidth}{0pt}
  \renewcommand{\footrulewidth}{0pt}
}

% ============================================================================
%                        FORMAT JUDUL BAB & SUBBAB
% ============================================================================
% Penomoran Chapter dengan Romawi (BAB I, BAB II, dst)
\renewcommand{\thechapter}{\Roman{chapter}}
\usepackage{titlesec}
\usepackage{etoolbox}

% Font untuk heading utama (14pt, bold)
\newcommand{\HeadingFont}{\fontsize{14pt}{21pt}\selectfont\bfseries}

% ==== FORMAT CHAPTER (BAB) ====
% Contoh hasil: "BAB I" (baris 1), "PENDAHULUAN" (baris 2)
\titleformat{\chapter}[display]
  {\normalfont\HeadingFont\centering}  % Format: 14pt, bold, tengah
  {\MakeUppercase{BAB \thechapter}}    % Label: BAB I, BAB II, dst
  {0pt}                                % Jarak antara label dan judul
  {\HeadingFont\MakeUppercase}         % Format judul: uppercase, bold, 14pt

% Spacing Chapter: {left}{before-sep}{after-sep}
\titlespacing*{\chapter}
  {0pt}              % left margin
  {0pt}              % space before (dari atas halaman)
  {1.5\baselineskip} % space after (ke konten berikutnya)

% ==== FORMAT SECTION (Heading 1: 1.1, 1.2, dst) ====
\renewcommand{\thesection}{\arabic{chapter}.\arabic{section}}
\titleformat{\section}[block]
  {\normalfont\bfseries}  % Format: bold, 12pt
  {\thesection}           % Nomor section
  {1em}                   % Jarak antara nomor dan judul
  {}                      % Kode sebelum judul

% Spacing Section
\titlespacing*{\section}
  {0pt}              % left margin
  {1\baselineskip}   % space before
  {0.5\baselineskip} % space after

% ==== FORMAT SUBSECTION (Heading 2: 1.1.1, 1.1.2, dst) ====
\renewcommand{\thesubsection}{\arabic{chapter}.\arabic{section}.\arabic{subsection}}
\titleformat{\subsection}[block]
  {\normalfont\bfseries}  % Format: bold, 12pt
  {\thesubsection}        % Nomor subsection
  {1em}                   % Jarak antara nomor dan judul
  {}                      % Kode sebelum judul

% Spacing Subsection
\titlespacing*{\subsection}
  {0pt}              % left margin
  {1\baselineskip}   % space before
  {0.5\baselineskip} % space after

% ==== FORMAT SUBSUBSECTION (Heading 3: 1.1.1.1, dst) ====
\renewcommand{\thesubsubsection}{\arabic{chapter}.\arabic{section}.\arabic{subsection}.\arabic{subsubsection}}
\titleformat{\subsubsection}[block]
  {\normalfont\bfseries}  % Format: bold, 12pt
  {\thesubsubsection}     % Nomor subsubsection
  {1em}                   % Jarak antara nomor dan judul
  {}                      % Kode sebelum judul

% Spacing Subsubsection
\titlespacing*{\subsubsection}
  {0pt}              % left margin
  {1\baselineskip}   % space before
  {0.5\baselineskip} % space after

% ==== FORMAT PARAGRAPH (Heading 4) ====
% Paragraph digunakan untuk sub-heading yang lebih kecil (jarang digunakan)
\setcounter{secnumdepth}{4}  % Aktifkan penomoran hingga paragraph
\titleformat{\paragraph}[runin]
  {\normalfont\bfseries}  % Format: bold, inline dengan teks
  {\theparagraph}         % Nomor paragraph
  {1em}                   % Jarak antara nomor dan judul
  {}                      % Kode sebelum judul
  [.]                     % Tanda setelah judul (titik)

% Spacing Paragraph
\titlespacing*{\paragraph}
  {0pt}              % left margin
  {1\baselineskip}   % space before
  {0.5em}            % space after (horizontal karena runin)

% ============================================================================
%                          INDENTASI PARAGRAF
% ============================================================================
% Indent ~4 karakter (1.5em untuk Times 12pt)
\setlength{\parindent}{1.5em}

% ============================================================================
%                         KONFIGURASI DAFTAR ISI (TOC)
% ============================================================================
\usepackage{tocloft}

% Kedalaman TOC: tampilkan hingga 1.1.1 (subsubsection)
\setcounter{tocdepth}{3}
\setcounter{secnumdepth}{3}

% Custom format untuk BAB di TOC
\renewcommand{\cftchappresnum}{BAB~}  % Prefix "BAB" sebelum nomor
\renewcommand{\cftchapaftersnum}{}    % Tidak ada spasi setelah nomor
\setlength{\cftchapindent}{0em}       % Chapter dimulai dari margin kiri
\setlength{\cftchapnumwidth}{5em}     % Lebar untuk "BAB III"

% Format teks Chapter di TOC: bold dengan leader dots
\renewcommand{\cftchapfont}{\bfseries}
\renewcommand{\cftchappagefont}{\bfseries}
\renewcommand{\cftchapleader}{\cftdotfill{\cftdotsep}}

% Format SECTION di TOC - Normal (tidak bold)
\renewcommand{\cftsecfont}{\normalfont}
\renewcommand{\cftsecpagefont}{\normalfont}
\setlength{\cftsecindent}{2em}
\setlength{\cftsecnumwidth}{3em}

% Format SUBSECTION di TOC
\setlength{\cftsubsecindent}{5em}
\setlength{\cftsubsecnumwidth}{4em}

% Format SUBSUBSECTION di TOC
\setlength{\cftsubsubsecindent}{9em}
\setlength{\cftsubsubsecnumwidth}{5em}

% Jarak konsisten antar entri TOC
\setlength{\cftbeforechapskip}{0.5em}
\setlength{\cftbeforesecskip}{0em}
\setlength{\cftbeforesubsecskip}{0em}
\setlength{\cftbeforesubsubsecskip}{0em}

% Kepadatan titik leader dots (1 = rapat)
\renewcommand{\cftdotsep}{1}

% Nama-nama daftar dalam Bahasa Indonesia
\renewcommand{\contentsname}{DAFTAR ISI}
\renewcommand{\listfigurename}{DAFTAR GAMBAR}
\renewcommand{\listtablename}{DAFTAR TABEL}

% Format judul daftar (centered, 14pt bold)
\renewcommand{\cfttoctitlefont}{\hfill\HeadingFont}
\renewcommand{\cftaftertoctitle}{\hfill}
\renewcommand{\cftloftitlefont}{\hfill\HeadingFont}
\renewcommand{\cftafterloftitle}{\hfill}
\renewcommand{\cftlottitlefont}{\hfill\HeadingFont}
\renewcommand{\cftafterlottitle}{\hfill}
\renewcommand{\cftchapleader}{\cftdotfill{\cftdotsep}}
\renewcommand{\cftsecleader}{\cftdotfill{\cftdotsep}}
\setlength{\cftbeforetoctitleskip}{0pt}
\setlength{\cftaftertoctitleskip}{1\baselineskip}
\setlength{\cftbeforeloftitleskip}{0pt}
\setlength{\cftafterloftitleskip}{1\baselineskip}
\setlength{\cftbeforelottitleskip}{0pt}
\setlength{\cftafterlottitleskip}{1\baselineskip}

% ==== Standarisasi Daftar Gambar (LoF) ====
\setlength{\cftfignumwidth}{3.5em}
\setlength{\cftfigindent}{0em}
\renewcommand{\cftdotsep}{1}
\setlength{\cftbeforefigskip}{6pt}
\makeatletter
\renewcommand*{\l@figure}{\@dottedtocline{1}{0em}{3.5em}}
\makeatother
\renewcommand{\cftfigleader}{\cftdotfill{\cftdotsep}}
\renewcommand{\cftfigpresnum}{}
\renewcommand{\cftfigaftersnum}{\space}

% ==== Standarisasi Daftar Tabel (LoT) ====
\setlength{\cfttabnumwidth}{3.5em}
\setlength{\cfttabindent}{0em}
\setlength{\cftbeforetabskip}{6pt}
\makeatletter
\renewcommand*{\l@table}{\@dottedtocline{1}{0em}{3.5em}}
\makeatother
\renewcommand{\cfttableader}{\cftdotfill{\cftdotsep}}
\renewcommand{\cfttabpresnum}{}
\renewcommand{\cfttabaftersnum}{\space}

% ============================================================================
%                           DAFTAR PERSAMAAN
% ============================================================================
\newcommand{\listequationname}{DAFTAR PERSAMAAN}
\newcommand{\eqnumberwithparen}{(\arabic{chapter}.\arabic{equation})}
\newlistof{myequations}{equ}{\listequationname}
\newcommand{\myequations}[1]{%
  % Menambahkan entri ke daftar persamaan dengan format (x.y) Keterangan
  \addcontentsline{equ}{myequations}{\protect\numberline{\eqnumberwithparen\enspace}#1}\par}
\setlength{\cftmyequationsnumwidth}{4em}
\providecommand{\cftmyequationstitlefont}{}
\providecommand{\cftaftermyequationstitle}{}
\renewcommand{\cftmyequationstitlefont}{\hfill\HeadingFont}
\renewcommand{\cftaftermyequationstitle}{\hfill}
\newcommand{\printlistequations}{%
  \RenderListPage[\addcontentsline{toc}{chapter}{\MakeUppercase{\listequationname}}]{\listequationname}{equ}}

% ============================================================================
%                         DAFTAR ALGORITMA/KODE
% ============================================================================
\newcommand{\listalgorithmname}{DAFTAR KODE DAN ALGORITMA}
\newcommand{\printlistalgorithms}{%
  \RenderListPageNoSpace[\addcontentsline{toc}{chapter}{\MakeUppercase{\listalgorithmname}}]{\listalgorithmname}{loa}%
}

% ============================================================================
%                           DAFTAR LAMPIRAN
% ============================================================================
\newcommand{\listappendixname}{DAFTAR LAMPIRAN}
\newlistof{myappendices}{apx}{\listappendixname}
\newcommand{\myappendix}[2]{%
\addcontentsline{apx}{myappendices}{\protect\numberline{Lampiran #1\enspace}#2}\par}
\setlength{\cftmyappendicesnumwidth}{7em}
\providecommand{\cftmyappendicestitlefont}{}
\providecommand{\cftaftermyappendicestitle}{}
\renewcommand{\cftmyappendicestitlefont}{\hfill\HeadingFont}
\renewcommand{\cftaftermyappendicestitle}{\hfill}
\newcommand{\printlistappendices}{%
  \RenderListPage[\addcontentsline{toc}{chapter}{\MakeUppercase{\listappendixname}}]{\listappendixname}{apx}}

% ============================================================================
%                      DAFTAR ISTILAH (GLOSARIUM)
% ============================================================================
\newcommand{\listtermname}{DAFTAR ISTILAH}
\newcommand{\myterm}[2]{\item[#1] #2}
\newcommand{\printlistterms}{%
  \cleardoublepage
  \phantomsection
  \thispagestyle{plain}%
  \setstretch{1.5}%
  \addcontentsline{toc}{chapter}{\MakeUppercase{\listtermname}}%
  {\HeadingFont\centering \MakeUppercase{\listtermname}\par}\vspace{\baselineskip}%
  \begin{description}[font=\bfseries, leftmargin=4cm, style=multiline]
    % chtex-lint: disable
% ============================================================================
%                           DAFTAR ISTILAH
% ============================================================================
% File ini berisi daftar istilah/terminologi yang digunakan dalam tesis
%
% PETUNJUK:
% - Gunakan command \myterm{Istilah}{Definisi} untuk menambahkan istilah
% - Istilah akan ditampilkan dalam format: Istilah ..... Definisi
% - Urutkan istilah secara alfabetis untuk memudahkan pembaca
% - Definisi harus jelas dan ringkas
%
% FORMAT:
% \myterm{Nama Istilah}{Penjelasan/definisi istilah tersebut}
%
% CONTOH:
% \myterm{AI}{Artificial Intelligence, kecerdasan buatan}
% \myterm{Machine Learning}{Pembelajaran mesin}
% ============================================================================

% ============================================================================
% ISI DAFTAR ISTILAH - TAMBAHKAN ISTILAH DI BAWAH INI
% ============================================================================
% PETUNJUK: Hapus contoh di bawah dan ganti dengan istilah tesis Anda
% Urutkan secara alfabetis (A-Z)
% ============================================================================

% --- Contoh Istilah (Hapus atau modifikasi sesuai kebutuhan) ---

\myterm{{Istilah 1}}{{Definisi atau penjelasan istilah 1. Tuliskan penjelasan yang jelas dan ringkas tentang istilah ini.}}

\myterm{{Istilah 2}}{{Definisi atau penjelasan istilah 2. Tuliskan penjelasan yang jelas dan ringkas tentang istilah ini.}}

\myterm{{Istilah 3}}{{Definisi atau penjelasan istilah 3. Tuliskan penjelasan yang jelas dan ringkas tentang istilah ini.}}

% --- Tambahkan istilah lainnya di bawah ini ---
% \myterm{Nama Istilah}{Definisi istilah}

  \end{description}
}

% ============================================================================
%                    GAMBAR, TABEL, DAN CAPTION
% ============================================================================
\usepackage{graphicx}
\usepackage{pdfpages}  % Untuk \includepdf pada lampiran
\usepackage{caption}

% --- TikZ untuk diagram dan flowchart ---
\usepackage{tikz}
\usepackage{pgfplots}
\pgfplotsset{compat=1.18}
\usetikzlibrary{shapes.geometric, shapes.misc, arrows.meta, positioning, calc, decorations.pathreplacing, backgrounds}

% ==== GAYA CAPTION INDONESIA ====
% Format Gambar: "Gambar x.y: Caption" (tidak bold)
\captionsetup[figure]{
  name=Gambar,                  % Bahasa Indonesia: "Gambar"
  labelsep=colon,               % Gambar 1.1: Caption text
  font=small,                   % Font lebih kecil untuk caption
  position=bottom,              % Caption di bawah gambar
  skip=6pt                      % Jarak antara gambar dan caption
}

% Format Tabel: "Tabel x.y: Caption" (tidak bold)
\captionsetup[table]{
  name=Tabel,                   % Bahasa Indonesia: "Tabel"
  labelsep=colon,               % Tabel 1.1: Caption Text
  font=small,                   % Font lebih kecil untuk caption
  position=top,                 % Caption di atas tabel
  skip=6pt,                     % Jarak antara caption dan tabel
  justification=centering       % Caption di tengah
}

% Pastikan amsmath dimuat untuk kontrol persamaan
\usepackage{amsmath}

% ============================================================================
%                         PSEUDOCODE/ALGORITMA
% ============================================================================
\usepackage[ruled,vlined,linesnumbered,resetcount]{algorithm2e}
\SetKwInput{KwInput}{Input}
\SetKwInput{KwOutput}{Output}
\SetKw{KwAnd}{and}
\SetKw{KwOr}{or}
\SetAlgoCaptionSeparator{.}
\SetAlgoNlRelativeSize{-1}
\SetAlFnt{\small}
\SetKwComment{Comment}{$\triangleright$ }{}

% Penomoran algoritme per bab: x.y (IEEE style)
\renewcommand{\thealgocf}{\arabic{chapter}.\arabic{algocf}}
\makeatletter
\@addtoreset{algocf}{chapter}
\makeatother

% Nama float - Gaya Indonesia
\SetAlgorithmName{Algoritma}{Algoritma}{DAFTAR KODE DAN ALGORITMA}

% ============================================================================
%                         KODE SUMBER (SOURCE CODE)
% ============================================================================
\usepackage{listings}

% --- Definisi Warna untuk Syntax Highlighting ---
\definecolor{codegreen}{rgb}{0,0.6,0}
\definecolor{codegray}{rgb}{0.5,0.5,0.5}
\definecolor{codepurple}{rgb}{0.58,0,0.82}
\definecolor{codeblue}{rgb}{0.13,0.29,0.53}
\definecolor{codeorange}{rgb}{0.8,0.4,0}
\definecolor{backcolour}{rgb}{0.97,0.97,0.97}

% --- Style Default untuk Kode ---
\lstdefinestyle{defaultstyle}{
    backgroundcolor=\color{backcolour},
    commentstyle=\color{codegreen}\itshape,
    keywordstyle=\color{codeblue}\bfseries,
    numberstyle=\tiny\color{codegray},
    stringstyle=\color{codeorange},
    basicstyle=\ttfamily\small,
    breakatwhitespace=false,
    breaklines=true,
    captionpos=b,
    keepspaces=true,
    numbers=left,
    numbersep=8pt,
    showspaces=false,
    showstringspaces=false,
    showtabs=false,
    tabsize=2,
    frame=single,
    framerule=0.5pt,
    rulecolor=\color{codegray},
    xleftmargin=2em,
    framexleftmargin=1.5em
}

% --- Style khusus Python ---
\lstdefinestyle{pythonstyle}{
    style=defaultstyle,
    language=Python,
    morekeywords={self, True, False, None, as, with, async, await, yield, lambda},
    emph={def, class, return, import, from, if, elif, else, for, while, try, except, finally, raise, pass, break, continue, and, or, not, in, is},
    emphstyle=\color{codepurple}\bfseries
}

% --- Style khusus JavaScript (definisi manual karena tidak built-in) ---
\lstdefinelanguage{JavaScript}{
    keywords={break, case, catch, continue, debugger, default, delete, do, else, finally, for, function, if, in, instanceof, new, return, switch, this, throw, try, typeof, var, void, while, with, class, const, enum, export, extends, import, super, implements, interface, let, package, private, protected, public, static, yield, async, await},
    morecomment=[l]{//},
    morecomment=[s]{/*}{*/},
    morestring=[b]',
    morestring=[b]",
    morestring=[b]`,
    sensitive=true
}

\lstdefinestyle{javascriptstyle}{
    style=defaultstyle,
    language=JavaScript,
    keywordstyle=\color{codeblue}\bfseries,
    commentstyle=\color{codegreen}\itshape,
    stringstyle=\color{codeorange}
}

% --- Style khusus Java ---
\lstdefinestyle{javastyle}{
    style=defaultstyle,
    language=Java,
    morekeywords={String, System, out, println},
    emph={public, private, protected, static, void, class, interface, extends, implements, new, return, if, else, for, while, try, catch, finally, throw},
    emphstyle=\color{codepurple}\bfseries
}

% Set style default
\lstset{style=defaultstyle}

% Penomoran listing per bab
\AtBeginDocument{
    \renewcommand{\thelstlisting}{\arabic{chapter}.\arabic{lstlisting}}
}
\makeatletter
\@addtoreset{lstlisting}{chapter}
\makeatother

% Nama caption dalam Bahasa Indonesia
\renewcommand{\lstlistingname}{Kode}
\renewcommand{\lstlistlistingname}{DAFTAR KODE SUMBER}

% ============================================================================
%                            TABEL (IEEE STYLE)
% ============================================================================
\usepackage{booktabs}        % Garis horizontal tabel berkualitas
\usepackage{siunitx}         % Perataan angka dan format numerik

% Konfigurasi siunitx
\sisetup{
  detect-weight=true,
  group-separator={,},
  group-minimum-digits=4,
  table-format=3.1,
}

% Ketebalan garis tabel dan spasi baris
\setlength{\arrayrulewidth}{0.3pt}
\renewcommand{\arraystretch}{1.15}

% ============================================================================
%                     SISTEM PENOMORAN (IEEE STYLE)
% ============================================================================
% Gambar: Angka Arab (1.1, 1.2, dst)
\numberwithin{figure}{chapter}
\renewcommand{\thefigure}{\arabic{chapter}.\arabic{figure}}

% Tabel: Angka Arab per bab (1.1, 1.2, 2.1, dst)
\numberwithin{table}{chapter}
\renewcommand{\thetable}{\arabic{chapter}.\arabic{table}}

% Persamaan: Angka Arab dengan tanda kurung (1.1, 1.2)
\numberwithin{equation}{chapter}
\renewcommand{\theequation}{\arabic{chapter}.\arabic{equation}}

% ============================================================================
%                          BIBLIOGRAFI (IEEE STYLE)
% ============================================================================
\usepackage{csquotes}
\usepackage[backend=biber,style=ieee,maxbibnames=99,minbibnames=99,sorting=none,isbn=false,url=false,date=year]{biblatex}
\addbibresource{references.bib}

% --- Suppress Biblatex Warnings ---
\usepackage{silence}
\WarningFilter{biblatex}{Using fallback definition for}
\WarningFilter{biblatex}{Language 'bahasai' not supported}
\WarningFilter{biblatex}{Bibliography string}
\WarningFilter{font}{Missing character}

% --- Fix Biblatex Localization untuk Bahasa Indonesia ---
\DeclareQuoteAlias{english}{bahasai}

% ===== HAPUS SEMUA FORMAT BOLD DARI BIBLIOGRAFI =====
\renewcommand*{\mkbibnamefamily}[1]{#1}
\renewcommand*{\mkbibnamegiven}[1]{#1}
\renewcommand*{\mkbibnameprefix}[1]{#1}
\renewcommand*{\mkbibnamesuffix}[1]{#1}
\renewcommand*{\mkbibacro}[1]{#1}
\renewcommand*{\mkbibemph}[1]{\textit{#1}}
\renewcommand*{\mkbibquote}[1]{``#1''}

\renewcommand*{\mkbibbold}[1]{#1}
\renewcommand*{\mkbibitalic}[1]{\textit{#1}}

\renewcommand*{\finalnamedelim}{\addspace and\space}
\renewcommand*{\multinamedelim}{\addcomma\space}

% Format field bibliografi
\DeclareFieldFormat{labelnumberwidth}{[#1]}
\DeclareFieldFormat{shorthandwidth}{#1}
\DeclareFieldFormat*{citetitle}{#1}
\DeclareFieldFormat*{title}{``#1''}
\DeclareFieldFormat[article,periodical]{volume}{#1}
\DeclareFieldFormat[article,periodical]{number}{(#1)}
\DeclareFieldFormat{pages}{pp\adddot\space#1}
\DeclareFieldFormat{journaltitle}{\textit{#1}}
\DeclareFieldFormat{booktitle}{\textit{#1}}
\DeclareFieldFormat{maintitle}{\textit{#1}}

% Format tanggal: hanya tahun, tanpa bold
\DeclareFieldFormat{date}{#1}
\DeclareFieldFormat{year}{#1}
\DeclareFieldFormat{urldate}{#1}

% Hilangkan bulan dan hari dari tampilan
\AtEveryBibitem{%
  \clearfield{note}%
  \clearfield{month}%
  \clearfield{day}%
}

% Override bibmacro
\renewbibmacro{in:}{%
  \printtext{in\addspace}%
}

% Punctuation
\renewcommand*{\newunitpunct}{\addcomma\space}
\renewcommand*{\finentrypunct}{\addperiod}

% Format field lainnya
\DeclareFieldFormat{publisher}{#1}
\DeclareFieldFormat{location}{#1}
\DeclareFieldFormat{chapter}{#1}
\DeclareFieldFormat{type}{#1}
\DeclareFieldFormat{version}{#1}
\DeclareFieldFormat{note}{#1}

% ============================================================================
%                              HYPERREF
% ============================================================================
\usepackage[hidelinks]{hyperref}

% ============================================================================
%                          CLEVEREF (CROSS-REFERENCE)
% ============================================================================
% HARUS dimuat SETELAH hyperref
\usepackage[capitalize,noabbrev]{cleveref}

% Konfigurasi nama Indonesia untuk cleveref
\crefname{figure}{Gambar}{Gambar}
\Crefname{figure}{Gambar}{Gambar}
\crefname{table}{Tabel}{Tabel}
\Crefname{table}{Tabel}{Tabel}
\crefname{equation}{Persamaan}{Persamaan}
\Crefname{equation}{Persamaan}{Persamaan}
\crefname{algocf}{Algoritma}{Algoritma}
\Crefname{algocf}{Algoritma}{Algoritma}

% Untuk custom code counter
\crefname{codenumber}{Algoritma}{Algoritma}
\Crefname{codenumber}{Algoritma}{Algoritma}

% ============================================================================
%                    VARIABEL METADATA TESIS
% ============================================================================
% PETUNJUK: Ganti placeholder {} dengan data Anda
%
% Contoh pengisian:
% \newcommand{\ThesisTitle}{JUDUL TESIS ANDA DENGAN HURUF KAPITAL}
% \newcommand{\AuthorName}{NAMA LENGKAP ANDA}
% ============================================================================

% --- Data Judul dan Penulis ---
% {Judul Tesis Lengkap}: Judul tesis dalam huruf kapital
\newcommand{\ThesisTitle}{{Judul Tesis Lengkap}}

% {NAMA LENGKAP PENULIS}: Nama lengkap dalam huruf kapital
\newcommand{\AuthorName}{{NAMA LENGKAP PENULIS}}

% {Nama Lengkap Penulis}: Nama lengkap dengan huruf kapital di awal kata
\newcommand{\AuthorNameB}{{Nama Lengkap Penulis}}

% {NIM}: Nomor Induk Mahasiswa, contoh: P31.2024.02610
\newcommand{\StudentID}{{NIM}}

% --- Data Program Studi (Tidak perlu diubah) ---
\newcommand{\ProgramName}{PROGRAM PASCASARJANA\\MAGISTER TEKNIK INFORMATIKA\\UNIVERSITAS DIAN NUSWANTORO}

% --- Data Lokasi dan Waktu ---
% {Kota}: Kota tempat pembuatan tesis
\newcommand{\City}{{Kota}}

% {Tahun}: Tahun pembuatan tesis
\newcommand{\Year}{{Tahun}}

% {Tujuan Tesis}: Kalimat tujuan tesis
\newcommand{\ThesisPurpose}{Tesis diajukan sebagai salah satu syarat untuk memperoleh gelar Magister Komputer}

% {Alamat Penulis}: Alamat tetap penulis
\newcommand{\AuthorAddress}{{Alamat Penulis}}

% --- Data Pembimbing ---
% {Nama Pembimbing Utama}: Nama lengkap dengan gelar
% Contoh: Prof. Dr. Nama Lengkap, S.T., M.Kom.
\newcommand{\SupervisorOne}{{Nama Pembimbing Utama}}

% {NIDN Pembimbing Utama}: Nomor Induk Dosen Nasional
\newcommand{\SupervisorOneNIDN}{{NIDN Pembimbing Utama}}

% {Nama Pembimbing Kedua}: Nama lengkap dengan gelar
\newcommand{\SupervisorTwo}{{Nama Pembimbing Kedua}}

% {NIDN Pembimbing Kedua}: Nomor Induk Dosen Nasional
\newcommand{\SupervisorTwoNIDN}{{NIDN Pembimbing Kedua}}

% --- Data Dewan Penguji ---
% {Nama Penguji 1}: Nama lengkap dengan gelar
\newcommand{\ExaminerOne}{{Nama Penguji 1}}

% {Nama Penguji 2}: Nama lengkap dengan gelar
\newcommand{\ExaminerTwo}{{Nama Penguji 2}}

% {Nama Dekan}: Nama lengkap dengan gelar
\newcommand{\DeanName}{{Nama Dekan}}

% {Nama Ketua Penguji}: Nama lengkap dengan gelar
\newcommand{\ChairExaminer}{{Nama Ketua Penguji}}


% ============================================================================
%                 COMMAND RENDER HALAMAN DAFTAR
% ============================================================================
\makeatletter
\newcommand{\RenderListPage}[3][]{%
  \cleardoublepage
  \phantomsection
  \thispagestyle{plain}%
  \setstretch{1.5}%
  #1%
  {\HeadingFont\centering \MakeUppercase{#2}\par}\vspace{\baselineskip}%
  \noindent\hfill\textbf{Halaman}\par\vspace{0.5\baselineskip}%
  \begingroup\setstretch{1.5}\@starttoc{#3}\endgroup\par}

% Command khusus untuk LoF dan LoT yang disable addvspace
\newcommand{\RenderListPageNoSpace}[3][]{%
  \cleardoublepage
  \phantomsection
  \thispagestyle{plain}%
  \setstretch{1.5}%
  #1%
  {\HeadingFont\centering \MakeUppercase{#2}\par}\vspace{\baselineskip}%
  \noindent\hfill\textbf{Halaman}\par\vspace{0.5\baselineskip}%
  \begingroup
  \setstretch{1.5}
  \let\addvspace\@gobble
  \@starttoc{#3}%
  \endgroup\par}
\makeatother

% ============================================================================
%                        FORMAT LAMPIRAN
% ============================================================================
\renewcommand{\appendixname}{LAMPIRAN}

% Override format chapter untuk lampiran
\makeatletter
\g@addto@macro\appendix{%
  \renewcommand{\chaptername}{\appendixname}%
  \renewcommand{\thechapter}{\Alph{kchapter}}%
  % Format chapter title untuk lampiran (hanya "LAMPIRAN A")
  \titleformat{\chapter}[display]
  {\normalfont\fontsize{14pt}{16pt}\selectfont\bfseries\centering}
  {\MakeUppercase{\appendixname~\thechapter}}
  {0pt}
  {\MakeUppercase}
}
\makeatother

% ============================================================================
%                              MULAI DOKUMEN
% ============================================================================
\begin{document}
\selectlanguage{bahasai}

% ############################################################################
%                         BAGIAN I: FRONTMATTER
%                    (Penomoran halaman: Romawi kecil)
% ############################################################################
% CATATAN: Halaman frontmatter menggunakan penomoran romawi kecil (i, ii, iii)
\pagenumbering{roman}

% ============================================================================
% HALAMAN JUDUL & PENGESAHAN
% ============================================================================
% PETUNJUK: Edit file-file berikut di folder frontmatter/
%   - cover.tex             : Halaman sampul depan
%   - titlepage.tex         : Halaman judul dalam
%   - pengesahan_status.tex : Halaman pengesahan status tesis
%   - pernyataan_penulis.tex: Halaman pernyataan keaslian
%   - persetujuan_tesis.tex : Halaman persetujuan sebelum sidang
%   - pengesahan_tesis.tex  : Halaman pengesahan setelah sidang
% ============================================================================
% chtex-lint: disable
% ============================================================================
%                           HALAMAN SAMPUL (COVER)
% ============================================================================
% Halaman sampul depan tesis, dicetak pada kertas linen berwarna orange
%
% PETUNJUK:
% - File ini menggunakan variabel dari main.tex
% - Pastikan variabel di main.tex sudah diisi dengan benar
% - Logo UDINUS: documents/udinus-logo.png (pastikan file ada)
%
% VARIABEL YANG DIGUNAKAN:
% - \ThesisTitle    : Judul tesis (dari main.tex)
% - \AuthorNameB    : Nama penulis huruf kapital di awal (dari main.tex)
% - \StudentID      : NIM penulis (dari main.tex)
% - \City           : Kota (dari main.tex)
% - \Year           : Tahun (dari main.tex)
% ============================================================================

% Uncomment baris berikut untuk background orange (opsional)
% \pagecolor{orange!30}

\thispagestyle{empty} % Sembunyikan nomor halaman pada cover

\begin{center}
  \color{darkred}

  \vspace*{2pt}

  % Label "TESIS" - 14pt Bold
  {\fontsize{14pt}{21pt}\selectfont\bfseries TESIS\par}

  \vspace{1cm}

  % Judul Tesis - 16pt Bold, Uppercase
  {\fontsize{16pt}{24pt}\selectfont\bfseries\MakeUppercase{\ThesisTitle}\par}

  \vspace{2cm}

  % Penulis dan NIM
  {\color{darkred}Oleh:}\\[0.5cm]
  {\color{darkred}\bfseries \AuthorNameB}\\[0.25cm]
  {\color{darkred}\StudentID}\\[2cm]

  % Logo UDINUS
  \includegraphics[width=4cm]{./documents/udinus-logo.png}\\[0.5cm]

  \vspace{1.0cm}

  % Informasi Program Studi - 14pt Bold
  {\fontsize{14pt}{22pt}\selectfont\bfseries
      PROGRAM PASCASARJANA\\[-0.5em]
      MAGISTER TEKNIK INFORMATIKA\\[-0.5em]
      UNIVERSITAS DIAN NUSWANTORO\\[-0.5em]
      \City\\[-0.5em]
      \Year
    \par}

\end{center}

% Uncomment untuk reset background setelah menggunakan \pagecolor
% \nopagecolor

% Pindah ke halaman baru (titlepage di halaman berikutnya)
\clearpage
                   % Sampul (kertas linen orange)
% chtex-lint: disable
% ============================================================================
%                         HALAMAN JUDUL (TITLE PAGE)
% ============================================================================
% Halaman judul dalam tesis, dicetak pada kertas HVS putih
%
% PETUNJUK:
% - File ini menggunakan variabel dari main.tex
% - Pastikan variabel di main.tex sudah diisi dengan benar
% - Logo UDINUS: documents/udinus-logo.png (pastikan file ada)
%
% VARIABEL YANG DIGUNAKAN:
% - \ThesisTitle    : Judul tesis (dari main.tex)
% - \ThesisPurpose  : Tujuan tesis (dari main.tex)
% - \AuthorName     : Nama penulis huruf kapital (dari main.tex)
% - \StudentID      : NIM penulis (dari main.tex)
% - \City           : Kota (dari main.tex)
% - \Year           : Tahun (dari main.tex)
% ============================================================================

\thispagestyle{empty} % Sembunyikan nomor halaman pada title page

\begin{center}

  \vspace*{2pt}

  % Label "TESIS" - 14pt Bold
  {\fontsize{14pt}{21pt}\selectfont\bfseries TESIS\par}

  \vspace{1cm}

  % Judul Tesis - 16pt Bold, Uppercase
  {\fontsize{16pt}{24pt}\selectfont\bfseries\MakeUppercase{\ThesisTitle}\par}

  \vspace{1.5cm}

  % Tujuan Tesis - Ukuran normal
  {\normalsize\ThesisPurpose\par}

  \vspace{1cm}

  % Nama Penulis dan NIM - Ukuran normal
  {\normalsize\AuthorName\\[0.3em]NIM: \StudentID\par}

  \vspace{1.7cm}

  % Logo UDINUS
  \includegraphics[width=4cm]{documents/udinus-logo.png}\\[1cm]

  \vspace{2cm}

  % Informasi Program Studi - 14pt Bold
  {\fontsize{14pt}{22pt}\selectfont\bfseries
    PROGRAM PASCASARJANA\\[-0.5em]
    MAGISTER TEKNIK INFORMATIKA\\[-0.5em]
    UNIVERSITAS DIAN NUSWANTORO\\[-0.5em]
    \City\\[-0.5em]
    \Year
  \par}

\end{center}

% Pindah ke halaman baru (pengesahan di halaman berikutnya)
\clearpage
               % Halaman Judul (HVS putih)
% chtex-lint: disable
% ============================================================================
%                      PENGESAHAN STATUS TESIS
% ============================================================================
% Halaman pengesahan status kerahasiaan tesis
%
% PETUNJUK:
% - Halaman ini berisi pernyataan status kerahasiaan tesis
% - Pilih salah satu status: Sangat Rahasia, Rahasia, atau Biasa
% - Untuk menandai pilihan, ganti \fbox{\phantom{X}} dengan \fbox{$\checkmark$}
%   atau \fbox{\textbf{X}}
%
% VARIABEL YANG DIGUNAKAN:
% - \ThesisTitle        : Judul tesis (dari main.tex)
% - \AuthorName         : Nama penulis huruf kapital (dari main.tex)
% - \AuthorAddress      : Alamat tetap penulis (dari main.tex)
% - \SupervisorOne      : Nama pembimbing utama (dari main.tex)
% - \SupervisorOneNIDN  : NIDN pembimbing utama (dari main.tex)
% ============================================================================

\cleardoublepage
\phantomsection
\addcontentsline{toc}{chapter}{PENGESAHAN STATUS TESIS}
\thispagestyle{plain}

\begin{center}
% Logo UDINUS
\includegraphics[width=3cm]{documents/udinus-logo.png}

\vspace{0.2em}

% Nama Universitas
{\bfseries UNIVERSITAS DIAN NUSWANTORO}

\vspace{0.5em}

% Judul Halaman
{\bfseries PENGESAHAN STATUS TESIS}

\end{center}

\vspace{0.2cm}

% ============================================================================
% PERNYATAAN STATUS
% ============================================================================

Tesis dengan Judul \textbf{\ThesisTitle}, Saya \AuthorName. Mengijinkan Tesis Magister Komputer ini disimpan di Perpustakaan Universitas Dian Nuswantoro dengan syarat-syarat ketentuan sebagai berikut:

\begin{enumerate}
\item Tesis adalah hak milik Universitas Dian Nuswantoro
\item Perpustakaan Universitas Dian Nuswantoro dibenarkan membuat salinan untuk tujuan referensi saja.
\item Perpustakaan juga dibenarkan membuat salinan Tesis ini sebagai bahan pertukaran antarinstitusi pendidikan tinggi.
\end{enumerate}

\vspace{0.1em}

% ============================================================================
% PILIHAN STATUS KERAHASIAAN
% ============================================================================
% PETUNJUK: Tandai salah satu pilihan dengan mengganti \fbox{\phantom{X}}
%           menjadi \fbox{$\checkmark$} atau \fbox{\textbf{X}}
% ============================================================================

% Di bagian isi dokumen:
\noindent\begin{tabularx}{\textwidth}{@{\hspace{0.5cm}}l@{\hspace{0.5cm}}X}
\fbox{\phantom{X}} & \textbf{Sangat Rahasia} (Mengandung isi tentang keselamatan atau kepentingan Negara Republik Indonesia) \\[0.3cm]
\fbox{\phantom{X}} & \textbf{Rahasia} (Mengandung isi tentang kerahasiaan dari suatu organisasi/badan tempat penelitian Tesis ini dikerjakan) \\[0.3cm]
\fbox{\textbf{X}} & \textbf{Biasa}
\end{tabularx}

\vspace{1cm}

\noindent \hspace{7cm} Disahkan oleh:

\vspace{2cm}

% ============================================================================
% BAGIAN TANDA TANGAN
% ============================================================================

\noindent
\begin{tabular*}{\textwidth}{@{}p{4.5cm}@{\extracolsep{\fill}}p{9cm}@{}}
\AuthorNameB & \SupervisorOne \\[0.1em]
\cline{2-2}
\AuthorAddress & \textbf{Pembimbing Utama} \\[0.1em]
& NIDN \SupervisorOneNIDN
\end{tabular*}

       % Pengesahan Status Tesis
% chtex-lint: disable
% ============================================================================
%                        PERNYATAAN PENULIS
% ============================================================================
% Halaman pernyataan keaslian karya tesis
%
% PETUNJUK:
% - Halaman ini berisi pernyataan bahwa tesis adalah karya asli penulis
% - Penulis harus membubuhkan meterai dan tanda tangan
% - Isi tanggal saat menandatangani pernyataan
%
% VARIABEL YANG DIGUNAKAN:
% - \ThesisTitle    : Judul tesis (dari main.tex)
% - \AuthorName     : Nama penulis huruf kapital (dari main.tex)
% - \StudentID      : NIM penulis (dari main.tex)
% ============================================================================

\cleardoublepage
\phantomsection
\addcontentsline{toc}{chapter}{PERNYATAAN PENULIS}
\thispagestyle{plain}

\begin{center}
% Logo UDINUS
\includegraphics[width=3cm]{documents/udinus-logo.png}

\vspace{0.5cm}

% Nama Universitas
{\bfseries UNIVERSITAS DIAN NUSWANTORO}

\vspace{1cm}

% Judul Halaman
{\bfseries PERNYATAAN PENULIS}

\end{center}

\vspace{0.8cm}

% ============================================================================
% METADATA TESIS
% ============================================================================

\noindent
\begin{tabular}{@{}l@{\hspace{0.5cm}}l@{\hspace{0.5cm}}p{\dimexpr\textwidth-1cm-1.5cm\relax}@{}}
JUDUL & : & \textbf{\ThesisTitle} \\[0.3cm]
NAMA & : & \AuthorName \\[0.3cm]
NIM & : & \StudentID
\end{tabular}

\vspace{0.8cm}

% ============================================================================
% TEKS PERNYATAAN
% ============================================================================

\noindent
``Saya menyatakan dan bertanggungjawab dengan sebenarnya bahwa Tesis ini adalah hasil karya saya sendiri kecuali cutipan dan ringkasan yang masing-masing saya jelaskan sumbernya. Jika pada waktu selanjutnya ada pihak lain yang mengklaim bahwa Tesis ini sebagai karyanya, yang disertai dengan bukti-bukti yang cukup, maka saya bersedia untuk dibatalkan gelar Magister Komputer saya beserta segala hak dan kewajiban yang melekat pada gelar tersebut''.

\vspace{1.5cm}

% ============================================================================
% BAGIAN TANDA TANGAN
% ============================================================================
% PETUNJUK: Isi tanggal dan tempel meterai Rp 10.000 pada Kotak meterai
% ============================================================================
\begin{center}
\Kota, \makebox[3cm]{\dotfill}
\end{center}

\vspace{0.5cm}

\noindent
\hspace{4cm}% Geser meterai ke kiri
\fbox{%
\begin{minipage}[c][1cm][c]{1.3cm}
\centering
Meterai
\end{minipage}%
}\hfill% ttd

\vspace{0.5cm}

\begin{center}
\textbf{\AuthorName}\\[-0.5em]
\rule{5cm}{0.4pt}\\[-0.5em]
Penulis
\end{center}
      % Pernyataan Keaslian Karya
% chtex-lint: disable
% ============================================================================
%                        PERSETUJUAN TESIS
% ============================================================================
% Halaman persetujuan tesis sebelum sidang
%
% PETUNJUK:
% - Halaman ini ditandatangani oleh pembimbing sebelum sidang
% - Menandakan bahwa tesis layak untuk diujikan
% - Isi tanggal persetujuan pada bagian tanda tangan
%
% VARIABEL YANG DIGUNAKAN:
% - \ThesisTitle    : Judul tesis (dari main.tex)
% - \AuthorName     : Nama penulis huruf kapital (dari main.tex)
% - \StudentID      : NIM penulis (dari main.tex)
% - \SupervisorOne  : Nama pembimbing utama (dari main.tex)
% - \SupervisorTwo  : Nama pembimbing kedua (dari main.tex)
% ============================================================================

\cleardoublepage
\phantomsection
\addcontentsline{toc}{chapter}{PERSETUJUAN TESIS}
\thispagestyle{plain}

\begin{center}
% Logo UDINUS
\includegraphics[width=3cm]{documents/udinus-logo.png}

\vspace{0.5cm}

% Nama Universitas
{\bfseries UNIVERSITAS DIAN NUSWANTORO}

\vspace{1cm}

% Judul Halaman
{\bfseries PERSETUJUAN TESIS}

\end{center}

\vspace{0.8cm}

% ============================================================================
% METADATA TESIS
% ============================================================================

\noindent
\begin{tabular}{@{}l@{\hspace{0.5cm}}l@{\hspace{0.5cm}}p{\dimexpr\textwidth-1cm-1.5cm\relax}@{}}
JUDUL & : & \textbf{\ThesisTitle} \\[0.3cm]
NAMA & : & \AuthorName \\[0.3cm]
NIM & : & \StudentID
\end{tabular}

\vspace{1cm}

% ============================================================================
% TEKS PERSETUJUAN
% ============================================================================

\begin{center}
Tesis ini telah diperiksa dan disetujui,

\vspace{0.5cm}

{Kota}, \makebox[4cm]{\dotfill}  % Ganti {Kota} dengan kota Anda
\end{center}

\vspace{3cm}

% ============================================================================
% BAGIAN TANDA TANGAN PEMBIMBING
% ============================================================================
% PETUNJUK: Pembimbing menandatangani setelah tesis siap untuk diujikan
% ============================================================================

\noindent
\begin{minipage}[t]{0.49\textwidth}
\centering
{\footnotesize \underline{\SupervisorOne}}\\
{\footnotesize Pembimbing Utama}
\end{minipage}%
\hfill
\begin{minipage}[t]{0.49\textwidth}
\centering
{\footnotesize \underline{\SupervisorTwo}}\\
{\footnotesize Pembimbing Pendamping}
\end{minipage}
       % Persetujuan Tesis (sebelum ujian)
% chtex-lint: disable
% ============================================================================
%                        PENGESAHAN TESIS
% ============================================================================
% Halaman pengesahan tesis setelah sidang
%
% PETUNJUK:
% - Halaman ini ditandatangani setelah sidang tesis
% - Ditandatangani oleh: Dewan Penguji, Dekan, dan Ketua Penguji
% - Isi tanggal sidang pada teks pengesahan dan bagian tanda tangan
%
% VARIABEL YANG DIGUNAKAN:
% - \ThesisTitle    : Judul tesis (dari main.tex)
% - \AuthorName     : Nama penulis huruf kapital (dari main.tex)
% - \StudentID      : NIM penulis (dari main.tex)
% - \ExaminerOne    : Nama penguji 1 (dari main.tex)
% - \ExaminerTwo    : Nama penguji 2 (dari main.tex)
% - \DeanName       : Nama dekan (dari main.tex)
% - \ChairExaminer  : Nama ketua penguji (dari main.tex)
% ============================================================================

\cleardoublepage
\phantomsection
\addcontentsline{toc}{chapter}{PENGESAHAN TESIS}
\thispagestyle{plain}

\begin{center}
% Logo UDINUS
\includegraphics[width=3cm]{documents/udinus-logo.png}

\vspace{0.5cm}

% Nama Universitas
{\bfseries UNIVERSITAS DIAN NUSWANTORO}

\vspace{1cm}

% Judul Halaman
{\bfseries PENGESAHAN TESIS}

\end{center}

\vspace{0.8cm}

% ============================================================================
% METADATA TESIS
% ============================================================================

\noindent
\begin{tabular}{@{}l@{\hspace{0.5cm}}l@{\hspace{0.5cm}}p{\dimexpr\textwidth-1cm-1.5cm\relax}@{}}
JUDUL & : & \textbf{\ThesisTitle} \\[0.3cm]
NAMA & : & \AuthorName \\[0.3cm]
NIM & : & \StudentID
\end{tabular}

\vspace{0.8cm}

% ============================================================================
% TEKS PENGESAHAN
% ============================================================================
% PETUNJUK: Ganti {Tanggal Sidang} dengan tanggal sidang yang sebenarnya
%           Contoh: "25 Maret 2025"
% ============================================================================

\noindent
\hspace{0.5cm} Tesis ini telah diujikan dan dipertahankan di hadapan Dewan Penguji pada Sidang Tesis tanggal {Tanggal Sidang}. Menurut pandangan kami, Tesis ini memadai dari segi kualitas untuk tujuan penganugerahan gelar Magister Komputer (M.Kom.)

\vspace{0.8cm}

% ============================================================================
% BAGIAN TANDA TANGAN DEWAN PENGUJI
% ============================================================================
% PETUNJUK: Isi tanggal pengesahan
% ============================================================================

\begin{center}
{Kota}, \makebox[4cm]{\dotfill}  % Ganti {Kota} dengan kota Anda

\vspace{0.5cm}

Dewan Penguji:
\end{center}

\vspace{1.5cm}

% --- Baris Pertama: 2 Kolom Anggota Penguji ---
\noindent
\begin{minipage}[t]{0.49\textwidth}
\centering
{\small \underline{\ExaminerOne}}\\
{\small Anggota}
\end{minipage}%
\hfill
\begin{minipage}[t]{0.49\textwidth}
\centering
{\small \underline{\ExaminerTwo}}\\
{\small Anggota}
\end{minipage}

\vspace{3em}

% --- Baris Kedua: Mengetahui ---
\begin{center}
{\small Mengetahui,}
\end{center}

\vspace{3em}

% --- Baris Ketiga: 2 Kolom Dekan dan Ketua Penguji ---
\noindent
\begin{minipage}[t]{0.49\textwidth}
\centering
{\small \underline{\DeanName}}\\
{\small Dekan}
\end{minipage}%
\hfill
\begin{minipage}[t]{0.49\textwidth}
\centering
{\small \underline{\ChairExaminer}}\\
{\small Ketua Penguji}
\end{minipage}
        % Pengesahan Tesis (sesudah sidang)

% ============================================================================
% ABSTRAK & KATA PENGANTAR
% ============================================================================
% PETUNJUK: Edit file-file berikut di folder frontmatter/
%   - abstract_en.tex       : Abstract dalam Bahasa Inggris
%   - abstract_id.tex       : Abstrak dalam Bahasa Indonesia
%   - acknowledgments.tex   : Kata pengantar/ucapan terima kasih
% ============================================================================
% chtex-lint: disable
% ============================================================================
%                         ABSTRACT (BAHASA INGGRIS)
% ============================================================================
% Halaman abstrak dalam Bahasa Inggris
%
% PETUNJUK:
% - Maksimum 200 kata
% - Isi harus mencakup:
%   (1) Masalah utama dan ruang lingkup penelitian
%   (2) Metode yang digunakan
%   (3) Hasil yang diperoleh
%   (4) Kesimpulan utama dan saran
% - Keywords dipisahkan dengan titik koma (;)
%
% VARIABEL YANG DIGUNAKAN:
% - \ThesisTitle    : Judul tesis (dari main.tex)
% - \AuthorName     : Nama penulis (dari main.tex)
% ============================================================================

\chapter*{ABSTRACT}
\addcontentsline{toc}{chapter}{ABSTRACT}

% Header abstrak dengan judul dan nama penulis
\noindent\textbf{Title:} \ThesisTitle\\
\textbf{Author:} \AuthorName\\[0.75\baselineskip]

% ============================================================================
% ISI ABSTRAK - GANTI TEKS DI BAWAH INI DENGAN ABSTRAK ANDA
% ============================================================================
% PETUNJUK: Hapus teks placeholder di bawah dan ganti dengan abstrak tesis Anda
% Maksimum 200 kata, mencakup: masalah, metode, hasil, dan kesimpulan
% ============================================================================

{Isi abstrak dalam Bahasa Inggris di sini. Abstrak (maksimum 200 kata) memuat: (1) masalah utama dan ruang lingkup penelitian, (2) metode yang digunakan, (3) hasil yang diperoleh, dan (4) kesimpulan utama serta saran.}

% ============================================================================
% KEYWORDS - GANTI DENGAN KATA KUNCI TESIS ANDA
% ============================================================================
% PETUNJUK: Tuliskan 3-5 kata kunci yang relevan, dipisahkan dengan titik koma
% ============================================================================

\textbf{Keywords:} {keyword 1}; {keyword 2}; {keyword 3}; {keyword 4}; {keyword 5}.
             % Abstract (English)
% chtex-lint: disable
% ============================================================================
%                        ABSTRAK (BAHASA INDONESIA)
% ============================================================================
% Halaman abstrak dalam Bahasa Indonesia
%
% PETUNJUK:
% - Maksimum 200 kata
% - Isi harus mencakup:
%   (1) Masalah utama dan ruang lingkup penelitian
%   (2) Metode yang digunakan
%   (3) Hasil yang diperoleh
%   (4) Kesimpulan utama dan saran
% - Kata kunci dipisahkan dengan titik koma (;)
%
% VARIABEL YANG DIGUNAKAN:
% - \ThesisTitle    : Judul tesis (dari main.tex)
% - \AuthorName     : Nama penulis (dari main.tex)
% ============================================================================

\chapter*{ABSTRAK}
\addcontentsline{toc}{chapter}{ABSTRAK}

% Header abstrak dengan judul dan nama penulis
\noindent\textbf{Judul:} \ThesisTitle\\
\textbf{Penulis:} \AuthorName\\[0.75\baselineskip]

% ============================================================================
% ISI ABSTRAK - GANTI TEKS DI BAWAH INI DENGAN ABSTRAK ANDA
% ============================================================================
% PETUNJUK: Hapus teks placeholder di bawah dan ganti dengan abstrak tesis Anda
% Maksimum 200 kata, mencakup: masalah, metode, hasil, dan kesimpulan
% ============================================================================

{Isi abstrak dalam Bahasa Indonesia di sini. Abstrak (maksimum 200 kata) memuat: (1) masalah utama dan ruang lingkup penelitian, (2) metode yang digunakan, (3) hasil yang diperoleh, dan (4) kesimpulan utama serta saran.}

% ============================================================================
% KATA KUNCI - GANTI DENGAN KATA KUNCI TESIS ANDA
% ============================================================================
% PETUNJUK: Tuliskan 3-5 kata kunci yang relevan, dipisahkan dengan titik koma
% ============================================================================

\textbf{Kata kunci:} {kata kunci 1}; {kata kunci 2}; {kata kunci 3}; {kata kunci 4}; {kata kunci 5}.
             % Abstrak (Bahasa Indonesia)
% chtex-lint: disable
% ============================================================================
%                     KATA PENGANTAR (ACKNOWLEDGMENTS)
% ============================================================================
% Halaman kata pengantar/ucapan terima kasih
%
% PETUNJUK:
% - Tuliskan ucapan terima kasih kepada pihak yang membantu
% - Biasanya mencakup:
%   1. Tuhan Yang Maha Esa
%   2. Keluarga
%   3. Dosen pembimbing
%   4. Dosen penguji
%   5. Teman dan kolega
%   6. Pihak lain yang membantu
% - Akhiri dengan harapan tesis bermanfaat
%
% VARIABEL YANG DIGUNAKAN:
% - \ThesisTitle    : Judul tesis (dari main.tex)
% - \AuthorNameB    : Nama penulis huruf kapital di awal (dari main.tex)
% ============================================================================

\chapter*{KATA PENGANTAR}
\addcontentsline{toc}{chapter}{KATA PENGANTAR}

% ============================================================================
% ISI KATA PENGANTAR - SESUAIKAN DENGAN TESIS ANDA
% ============================================================================
% PETUNJUK: Sesuaikan isi kata pengantar dengan kondisi Anda
% ============================================================================

Tesis dengan judul \textbf{\ThesisTitle} ini dapat penulis selesaikan sesuai rencana karena dukungan dari berbagai pihak yang tidak ternilai besarnya. Oleh karena itu penulis menyampaikan terima kasih kepada:

\begin{enumerate}
    % --- Ucapan Terima Kasih Kepada Tuhan ---
    \item {Tuhan Yang Maha Esa yang telah memberikan rahmat dan karunia-Nya sehingga penulis dapat menyelesaikan tesis ini dengan baik.}

    % --- Ucapan Terima Kasih Kepada Keluarga ---
    \item {Keluarga tercinta yang telah memberikan dukungan moral, doa, dan motivasi yang tiada henti selama proses penulisan tesis ini.}

    % --- Ucapan Terima Kasih Kepada Pembimbing Utama ---
    % PETUNJUK: Ganti dengan nama pembimbing utama Anda
    \item {{Nama Pembimbing Utama}, selaku pembimbing utama yang telah memberikan arahan, masukan, dan bimbingan yang sangat berharga dalam penyusunan tesis ini.}

    % --- Ucapan Terima Kasih Kepada Pembimbing Kedua ---
    % PETUNJUK: Ganti dengan nama pembimbing kedua Anda
    \item {{Nama Pembimbing Kedua}, selaku pembimbing kedua yang telah meluangkan waktu dan memberikan perspektif berbeda serta saran konstruktif dalam penyempurnaan tesis ini.}

    % --- Ucapan Terima Kasih Kepada Pihak Lain ---
    % PETUNJUK: Tambahkan pihak lain yang membantu (validator, narasumber, dll)
    \item {Pihak-pihak yang membantu dalam pelaksanaan penelitian, khususnya dalam proses pengumpulan data dan validasi instrumen penelitian.}

    % --- Ucapan Terima Kasih Umum ---
    \item Semua pihak yang tidak dapat disebutkan satu per satu, yang telah memberikan bantuan, dukungan, dan kontribusi dalam penyelesaian tesis ini.
\end{enumerate}

Penulis menyadari adanya keterbatasan penelitian ini, maka kritik, saran, dan masukan yang membangun akan sangat membantu penulis dalam penelitian selanjutnya. Semoga tulisan ini dapat bermanfaat bagi ilmu pengetahuan dan pembaca.

% ============================================================================
% TANDA TANGAN PENULIS
% ============================================================================
% PETUNJUK: Isi tanggal saat tesis selesai
% ============================================================================

\vspace{1cm}
\begin{flushright}
{Kota}, \rule{2.5cm}{0.4pt}\\  % Ganti {Kota} dengan kota Anda
\vspace{3em}
\AuthorNameB
\end{flushright}
         % Kata Pengantar

% ============================================================================
% DAFTAR-DAFTAR
% ============================================================================
% PETUNJUK: Uncomment (hapus %) pada baris yang diperlukan
%           Comment (tambah %) pada baris yang belum diperlukan
% ============================================================================
% chtex-lint: disable
\cleardoublepage
\phantomsection
\addcontentsline{toc}{chapter}{DAFTAR ISI}
\thispagestyle{plain}

% Spasi 1.5
\setstretch{1.5}

% Judul DAFTAR ISI (centered, bold, 14pt)
{\HeadingFont\centering DAFTAR ISI\par}

\vspace{\baselineskip}

% Header kolom "Halaman" di kanan atas
\noindent\hfill\textbf{Halaman}

\vspace{0.5\baselineskip}

% Table of Contents (tanpa judul otomatis)
\makeatletter
\@starttoc{toc}
\makeatother
              % Daftar Isi
% Daftar Gambar
\RenderListPageNoSpace[\addcontentsline{toc}{chapter}{\MakeUppercase{\listfigurename}}]{\listfigurename}{lof}
      % Daftar Gambar
%% Daftar Tabel
\RenderListPageNoSpace[\addcontentsline{toc}{chapter}{\MakeUppercase{\listtablename}}]{\listtablename}{lot}
       % Daftar Tabel
\input{frontmatter/daftar_persamaan}        % Daftar Persamaan
%% Daftar Kode dan Algoritma
\printlistalgorithms
        % Daftar Kode dan Algoritma
\input{frontmatter/daftar_lampiran}         % Daftar Lampiran
\input{frontmatter/daftar_istilah_list}     % Daftar Istilah

% ############################################################################
%                         BAGIAN II: CHAPTERS (ISI)
%                      (Penomoran halaman: Angka Arab)
% ############################################################################
% CATATAN: Halaman isi menggunakan penomoran angka Arab (1, 2, 3)
\clearpage
\pagenumbering{arabic}

% ============================================================================
% BAB-BAB TESIS
% ============================================================================
% PETUNJUK: Edit file-file berikut di folder chapters/
%   - bab1_pendahuluan.tex      : Latar belakang, rumusan masalah, tujuan
%   - bab2_tinjauan_pustaka.tex : Landasan teori dan kajian pustaka
%   - bab3_metode_penelitian.tex: Metodologi penelitian
%   - bab4_pelaksanaan_hasil.tex: Hasil dan pembahasan
%   - bab5_penutup.tex          : Kesimpulan dan saran
%
% Uncomment (hapus %) pada baris yang sudah selesai ditulis
% ============================================================================
% chtex-lint: disable
\chapter{PENDAHULUAN}

\section{Latar Belakang}

Latar belakang tesis magister pada dasarnya adalah “cerita ilmiah” yang menjelaskan mengapa topik yang Anda teliti penting, apa yang bermasalah, dan di mana posisi penelitian Anda di antara karya-karya sebelumnya. Penulisannya mengikuti alur deduktif: mulai dari fenomena umum, mengerucut ke konteks yang lebih spesifik, lalu berakhir pada perumusan masalah dan kebutuhan penelitian.

Secara umum, struktur naratif latar belakang dapat diatur sebagai berikut:

Paragraf awal berisi gambaran besar fenomena atau tren di bidang yang Anda teliti (misalnya perkembangan AI, computer vision, sistem pendidikan, dan sebagainya), ditulis dengan kalimat pembuka yang kuat dan informatif.

Paragraf-paragraf berikutnya membawa pembaca ke konteks yang lebih sempit: kondisi nasional, institusional, atau domain tertentu yang menjadi fokus tesis, sekaligus menunjukkan adanya masalah nyata (misalnya kinerja sistem belum optimal, kualitas layanan masih rendah, atau dampak negatif yang belum tertangani). Di sini, Anda menjelaskan urgensi penelitian, yaitu mengapa masalah ini penting untuk segera dikaji, baik dari sisi teoretis maupun praktis.

Setelah itu, latar belakang perlu menonjolkan gap penelitian. Gap ini bisa berupa: topik yang belum banyak dikaji, konteks lokal yang belum disentuh, metode yang belum pernah diterapkan, atau kelemahan studi-studi terdahulu (misalnya data terbatas, metode kurang akurat, variabel penting belum dipertimbangkan). Paparkan gap secara eksplisit tetapi singkat, sehingga terlihat jelas bahwa masih ada ruang yang perlu diisi oleh penelitian Anda. Dari penjelasan urgensi dan gap inilah nantinya rumusan masalah dan tujuan penelitian menjadi terasa “wajar” dan logis.

Dari sisi aturan penulisan, gunakan bahasa ilmiah yang formal, objektif, dan konsisten (hindari “saya” atau opini pribadi yang tidak berdasar). Setiap klaim faktual idealnya didukung rujukan, tetapi detail teoritis dan teknis yang sangat spesifik sebaiknya disimpan untuk bab tinjauan pustaka. Latar belakang dibuat mengalir antar paragraf dengan penghubung yang jelas, tidak berupa poin-poin, dan biasanya memiliki panjang yang cukup untuk memberikan konteks yang matang (seringnya sekitar beberapa halaman dalam proporsi yang wajar terhadap keseluruhan bab pendahuluan). Dengan cara ini, latar belakang menjadi fondasi yang kokoh untuk memahami rumusan masalah, tujuan, dan keseluruhan rancangan penelitian.

\section{Rumusan Masalah}

Rumusan masalah dalam tesis magister menekankan penyajian berupa narasi deskriptif yang afirmatif dan terstruktur, bukan daftar pertanyaan, untuk menggambarkan secara jelas kesenjangan penelitian yang muncul dari latar belakang, sehingga menjadi jembatan logis menuju tujuan penelitian. Mulailah dengan mengintegrasikan elemen-elemen masalah secara koheren dalam paragraf utuh, yang mencakup identifikasi fenomena utama, variabel kunci, keterbatasan studi sebelumnya, serta implikasi dampaknya jika tidak ditangani, dengan panjang ideal 1-2 paragraf (sekitar 300-500 kata) yang mengalir dari umum ke spesifik tanpa menggunakan kata tanya. Pastikan narasi ini spesifik, terukur, dan dapat diuji melalui metode yang dipilih, sambil menunjukkan novelty penelitian seperti aplikasi konteks lokal, serta hindari pengulangan latar belakang dengan fokus pada fokus operasional masalah.

\textbf{Contoh Narasi Rumusan Masalah} \\
Model deep learning untuk deteksi objek pada citra dengan pencahayaan rendah masih menghadapi tantangan signifikan karena penurunan akurasi yang drastis, sehingga menyebabkan ketidakandalan pada aplikasi seperti pengawasan keamanan dan kendaraan otonom. Penurunan performa ini terutama disebabkan oleh kurangnya data pelatihan yang representatif dan ketidakmampuan algoritma saat menghadapi noise dan variasi intensitas cahaya. Selain itu, metode augmentasi data yang ada belum cukup efektif untuk memperbaiki kekurangan tersebut, sehingga diperlukan pengembangan teknik baru yang lebih robust dan adaptif terhadap kondisi pencahayaan ekstrem. Oleh karena itu, penelitian ini akan fokus pada pengembangan model Computer Vision yang dapat meningkatkan akurasi deteksi objek pada citra pencahayaan rendah dengan memanfaatkan teknik augmentasi data yang inovatif dan algoritma pembelajaran yang lebih adaptif.

Narasi seperti ini memperjelas fokus penelitian secara spesifik dan kontekstual di bidang Computer Vision, menciptakan landasan kuat untuk arah metodologi dan analisis yang akan dilakukan. Pemilihan bahasa harus formal dan objektif, serta menjelaskan urgensi penelitian tanpa menggunakan format tanya, untuk memperkuat argumen ilmiah tesis.

\section{Tujuan Penelitian}

Tujuan penelitian disusun untuk menjelaskan secara jelas apa yang ingin dicapai oleh tesis, sebagai turunan langsung dari rumusan masalah dan selaras dengan latar belakang. Penulisannya menggunakan kalimat pernyataan (bukan pertanyaan), bersifat spesifik, terukur, dan realistis dicapai dengan metode yang digunakan. Biasanya diawali dengan satu kalimat pembuka singkat, kemudian diikuti poin-poin tujuan dalam bentuk daftar agar mudah dibaca.

Secara umum, tata caranya bisa mengikuti pola berikut:
Pastikan setiap tujuan terkait langsung dengan rumusan masalah (satu rumusan masalah minimal satu tujuan). Bedakan bila perlu antara tujuan umum dan tujuan khusus (tujuan umum 1 kalimat, tujuan khusus dalam beberapa poin).
Gunakan kata kerja operasional yang jelas, seperti: menganalisis, menguji, mengukur, membandingkan, mengembangkan, merancang, mengevaluasi. Hindari tujuan yang terlalu luas atau normatif (misalnya “untuk memperbaiki pendidikan Indonesia”), dan fokus pada hal yang benar-benar dihasilkan tesis.

\section{Tujuan Penelitian}

Tujuan penelitian ini disusun sebagai turunan langsung dari rumusan masalah dan berfokus pada pengembangan serta evaluasi penerapan kecerdasan buatan dalam domain yang diteliti. Secara umum, tujuan penelitian dalam bidang kecerdasan buatan dapat dirumuskan sebagai berikut.

\begin{enumerate}
  \item Menganalisis kebutuhan sistem kecerdasan buatan pada domain tertentu (misalnya pendidikan, kesehatan, atau industri) sebagai dasar perancangan solusi berbasis AI.
  \item Merancang dan mengimplementasikan model kecerdasan buatan (seperti \textit{machine learning}, \textit{deep learning}, atau \textit{reinforcement learning}) yang sesuai dengan karakteristik data dan permasalahan yang dikaji.
  \item Mengevaluasi kinerja model kecerdasan buatan yang dikembangkan menggunakan metrik yang relevan (misalnya akurasi, presisi, recall, F1-score, atau metrik lain yang sesuai).
  \item Membandingkan hasil kinerja model yang diusulkan dengan metode atau pendekatan yang sudah ada untuk mengetahui keunggulan dan keterbatasan solusi yang dikembangkan.
  \item Memberikan rekomendasi pengembangan lebih lanjut dan potensi penerapan model kecerdasan buatan dalam konteks nyata sesuai domain aplikasi yang diteliti.
\end{enumerate}

Bagian Manfaat Penelitian menjelaskan “untungnya apa dan untuk siapa” dari hasil tesis yang Anda kerjakan. Isinya menjabarkan kontribusi yang diharapkan, baik untuk pengembangan ilmu (teoretis) maupun untuk pemecahan masalah nyata (praktis). Biasanya diawali satu paragraf pengantar singkat, lalu diikuti butir-butir manfaat agar mudah terbaca.

Secara umum, penataannya dapat dibuat seperti ini:

Jelaskan terlebih dahulu bahwa manfaat penelitian dibagi menjadi dua: manfaat teoretis dan manfaat praktis.

\begin{enumerate}
    \item \textbf{Manfaat Teoretis}: Menjelaskan kontribusi penelitian terhadap pengembangan ilmu, model, metode, atau kerangka konseptual di bidang kecerdasan buatan, seperti memberikan pemahaman baru tentang efektivitas algoritma tertentu atau mengembangkan kerangka kerja inovatif untuk analisis data.
    \item \textbf{Manfaat Praktis}: Merinci dampak konkret dan pihak-pihak yang diuntungkan secara langsung dari hasil penelitian, misalnya membantu guru dalam mengoptimalkan proses pembelajaran, mendukung pengembang sistem dalam menciptakan aplikasi yang lebih efisien, atau memberikan rekomendasi bagi institusi dan perusahaan dalam pengambilan keputusan berbasis data.
\end{enumerate}

Hindari klaim berlebihan; tulis manfaat yang realistis dan benar-benar terkait hasil yang mungkin dicapai oleh penelitian.
\section{Batasan Penelitian}

Agar penelitian ini terfokus dan dapat dilaksanakan secara realistis, maka ruang lingkup penelitian dibatasi oleh beberapa hal sebagai berikut.

\begin{enumerate}
  \item Penelitian ini hanya membahas permasalahan sesuai dengan rumusan masalah yang telah ditetapkan, sehingga tidak mencakup isu-isu lain di luar fokus kajian.
  \item Data yang digunakan dalam penelitian dibatasi pada sumber dan periode tertentu yang telah ditentukan sebelumnya, sehingga hasil yang diperoleh hanya berlaku dalam konteks tersebut.
  \item Variabel yang dianalisis dibatasi pada variabel-variabel yang telah didefinisikan secara operasional, sementara variabel lain yang berpotensi berpengaruh tidak dibahas secara mendalam.
  \item Metode yang digunakan dalam penelitian dibatasi pada pendekatan yang telah dipilih oleh peneliti, sehingga tidak melakukan perbandingan dengan seluruh metode atau teknik analisis yang tersedia.
  \item Hasil penelitian ini dibatasi pada kondisi dan asumsi yang digunakan selama proses pengumpulan dan pengolahan data, sehingga generalisasi temuan perlu dilakukan dengan mempertimbangkan keterbatasan tersebut.
\end{enumerate}
           % BAB I   - Pendahuluan
% chtex-lint: disable
\chapter{Menulis \LaTeX}

\section{Pengantar Penulisan \LaTeX}

Bagian ini memberikan pengenalan singkat penggunaan \LaTeX\ untuk penulisan ilmiah, meliputi penulisan teks tebal dan miring, pembuatan tabel, penyisipan gambar dengan sitasi, serta penulisan persamaan yang disertai sitasi. Contoh-contoh berikut dapat dijadikan acuan dasar dalam menulis dokumen tesis atau artikel ilmiah.

\subsection{Penulisan Teks Tebal dan Miring}

Penulisan teks tebal dan miring digunakan untuk menegaskan istilah tertentu atau memberi penekanan pada bagian penting.

\begin{itemize}
  \item Teks \textbf{tebal} menggunakan perintah \verb|\textbf{teks tebal}|.
  \item Teks \textit{miring} menggunakan perintah \verb|\textit{teks miring}|.
  \item Teks \emph{penekanan} menggunakan perintah \verb|\emph{teks penting}|.
\end{itemize}

Sebagai contoh: \textbf{istilah penting}, \textit{foreign term}, dan \emph{konsep utama} dapat digunakan untuk membedakan jenis penekanan dalam naskah.

\subsection{Penulisan Daftar (Itemize dan Enumerate)}

\LaTeX\ menyediakan beberapa lingkungan untuk membuat daftar. Dua yang paling umum digunakan adalah \verb|itemize| untuk daftar tanpa nomor (bullet points) dan \verb|enumerate| untuk daftar bernomor.

\textbf{Daftar Tanpa Nomor (Itemize)}

Gunakan lingkungan \verb|itemize| untuk membuat daftar dengan bullet:

\begin{verbatim}
\begin{itemize}
  \item Poin pertama
  \item Poin kedua
  \item Poin ketiga
\end{itemize}
\end{verbatim}

Hasil dari kode di atas:
\begin{itemize}
  \item Poin pertama yang menjelaskan konsep secara singkat.
  \item Poin kedua yang memberikan rincian atau contoh tambahan.
  \item Poin ketiga yang menegaskan hal-hal penting.
\end{itemize}

\textbf{Daftar Bernomor (Enumerate)}

Gunakan lingkungan \verb|enumerate| untuk membuat daftar dengan penomoran otomatis:

\begin{verbatim}
\begin{enumerate}
  \item Langkah pertama
  \item Langkah kedua
  \item Langkah ketiga
\end{enumerate}
\end{verbatim}

Hasil dari kode di atas:
\begin{enumerate}
  \item Langkah pertama dalam suatu prosedur atau tahapan analisis.
  \item Langkah kedua yang menjelaskan proses lanjutan.
  \item Langkah ketiga yang menggambarkan penyelesaian atau keluaran proses.
\end{enumerate}

\textbf{Daftar Bersarang (Nested List)}

Daftar dapat disarangkan hingga beberapa level:

\begin{enumerate}
  \item Kategori utama pertama
  \begin{enumerate}
    \item Sub-kategori 1.1
    \item Sub-kategori 1.2
  \end{enumerate}
  \item Kategori utama kedua
  \begin{itemize}
    \item Poin campuran dengan bullet
    \item Poin campuran lainnya
  \end{itemize}
\end{enumerate}

\subsection{Membuat Tabel dengan Format IEEE}

Tabel digunakan untuk menyajikan data secara terstruktur. Template ini menggunakan paket \texttt{booktabs} untuk menghasilkan garis horizontal berkualitas sesuai standar IEEE.

\textbf{Struktur Dasar Tabel}

Berikut adalah struktur dasar penulisan tabel:

\begin{verbatim}
\begin{table}[h]
  \centering
  \caption{Judul Tabel}
  \label{tab:label-tabel}
  \begin{tabular}{ccc}
    \toprule
    Kolom 1 & Kolom 2 & Kolom 3 \\
    \midrule
    Data 1  & Data 2  & Data 3 \\
    Data 4  & Data 5  & Data 6 \\
    \bottomrule
  \end{tabular}
\end{table}
\end{verbatim}

Perintah penting dalam format IEEE:
\begin{itemize}
  \item \verb|\toprule| -- garis horizontal tebal di atas tabel
  \item \verb|\midrule| -- garis horizontal tipis pemisah header dan data
  \item \verb|\bottomrule| -- garis horizontal tebal di bawah tabel
  \item \verb|\cmidrule{a-b}| -- garis horizontal parsial dari kolom a ke b
\end{itemize}

\textbf{Contoh Tabel Sederhana}

\begin{table}[h]
  \centering
  \caption{Contoh Tabel Sederhana dengan Format IEEE}
  \label{tab:contoh-tabel}
  \begin{tabular}{ccc}
    \toprule
    No & Parameter & Nilai \\
    \midrule
    1  & Contoh A & 0.85 \\
    2  & Contoh B & 0.90 \\
    3  & Contoh C & 0.78 \\
    \bottomrule
  \end{tabular}
\end{table}

\textbf{Contoh Tabel dengan Header Bertingkat}

\begin{table}[h]
  \centering
  \caption{Perbandingan Performa Model}
  \label{tab:performa-model}
  \begin{tabular}{lccc}
    \toprule
    \textbf{Model} & \textbf{Accuracy} & \textbf{Precision} & \textbf{Recall} \\
    \midrule
    Random Forest  & 0.92 & 0.89 & 0.91 \\
    SVM            & 0.88 & 0.85 & 0.87 \\
    Neural Network & 0.95 & 0.93 & 0.94 \\
    \midrule
    \textit{Rata-rata} & 0.92 & 0.89 & 0.91 \\
    \bottomrule
  \end{tabular}
\end{table}

Tabel \ref{tab:contoh-tabel} dan Tabel \ref{tab:performa-model} dapat dirujuk menggunakan perintah \verb|\ref{tab:label}|.

\subsection{Menyisipkan Gambar}

Gambar disisipkan menggunakan lingkungan \verb|figure| dengan perintah \verb|\includegraphics| dari paket \texttt{graphicx}.

\textbf{Struktur Dasar Gambar}

\begin{verbatim}
\begin{figure}[h]
  \centering
  \includegraphics[width=0.8\textwidth]{path/ke/gambar.png}
  \caption{Keterangan gambar.}
  \label{fig:label-gambar}
\end{figure}
\end{verbatim}

Opsi pengaturan ukuran gambar:
\begin{itemize}
  \item \verb|width=0.8\textwidth| -- lebar 80\% dari lebar teks
  \item \verb|height=5cm| -- tinggi tetap 5 cm
  \item \verb|scale=0.5| -- skala 50\% dari ukuran asli
  \item \verb|width=\linewidth| -- lebar penuh satu baris
\end{itemize}

\textbf{Contoh Penyisipan Gambar}

\begin{figure}[h]
  \centering
  \includegraphics[width=0.6\textwidth]{documents/Iris_dataset.png}
  \caption{Contoh dataset Iris yang diadaptasi dari penelitian sebelumnya \cite{Zhou2025}.}
  \label{fig:contoh-gambar}
\end{figure}

Gambar~\ref{fig:contoh-gambar} menunjukkan struktur dataset Iris yang umum digunakan dalam penelitian machine learning. Gunakan \verb|Gambar~\ref{fig:label}| untuk merujuk gambar (tanda \verb|~| mencegah pemisahan baris).

\textbf{Opsi Penempatan Float}

Opsi penempatan dalam tanda kurung siku:
\begin{itemize}
  \item \verb|[h]| -- here, tempatkan di posisi saat ini jika memungkinkan
  \item \verb|[t]| -- top, tempatkan di bagian atas halaman
  \item \verb|[b]| -- bottom, tempatkan di bagian bawah halaman
  \item \verb|[p]| -- page, tempatkan di halaman khusus float
  \item \verb|[H]| -- memaksa penempatan tepat di posisi kode (memerlukan paket \texttt{float})
  \item \verb|[htbp]| -- kombinasi, biarkan \LaTeX\ memilih yang terbaik
\end{itemize}

\subsection{Menulis Persamaan}

Persamaan matematika dapat ditulis dalam mode \emph{display} menggunakan lingkungan \verb|equation| dan diberi label untuk dirujuk kembali.

\textbf{Persamaan dengan Penomoran}

\begin{verbatim}
\begin{equation}
  y = f(x) = ax^{2} + bx + c
  \label{eq:persamaan-kuadrat}
\end{equation}
\myequations{Deskripsi Persamaan}
\end{verbatim}

Contoh hasil:

\begin{equation}
  y = f(x) = ax^{2} + bx + c
  \label{eq:persamaan-kuadrat}
\end{equation}
\myequations{Persamaan Fungsi Kuadrat}

Persamaan \eqref{eq:persamaan-kuadrat} dapat dirujuk menggunakan \verb|\eqref{eq:label}| atau \verb|\ref{eq:label}|. Perintah \verb|\myequations{...}| digunakan untuk menambahkan persamaan ke Daftar Persamaan.

\textbf{Persamaan Tanpa Penomoran}

Untuk persamaan yang tidak perlu dirujuk, gunakan \verb|\[ ... \]|:

\[
  \nabla J(\theta) = \frac{\partial J(\theta)}{\partial \theta}
\]

\subsection{Sitasi dan Daftar Pustaka}

Sitasi dalam template ini menggunakan format IEEE dengan perintah \verb|\cite{key}|. Contoh: \verb|\cite{Zhou2025}| menghasilkan \cite{Zhou2025}.

Langkah kompilasi untuk memproses sitasi:
\begin{enumerate}
  \item \textbf{XeLaTeX} -- kompilasi pertama
  \item \textbf{Biber} -- memproses bibliografi
  \item \textbf{XeLaTeX} -- kompilasi kedua dan ketiga
\end{enumerate}

\subsection{Label dan Rujukan Silang}

Gunakan \verb|\label{...}| untuk memberi penanda pada elemen, lalu rujuk dengan \verb|\ref{...}| atau \verb|\cref{...}|.

\begin{table}[h]
  \centering
  \caption{Konvensi Penamaan Label}
  \label{tab:konvensi-label}
  \begin{tabular}{lll}
    \toprule
    \textbf{Elemen} & \textbf{Prefix} & \textbf{Contoh} \\
    \midrule
    Gambar    & \texttt{fig:} & \verb|\label{fig:arsitektur}| \\
    Tabel     & \texttt{tab:} & \verb|\label{tab:hasil}| \\
    Persamaan & \texttt{eq:}  & \verb|\label{eq:loss-function}| \\
    Algoritma & \texttt{alg:} & \verb|\label{alg:training}| \\
    Kode      & \texttt{lst:} & \verb|\label{lst:python-code}| \\
    \bottomrule
  \end{tabular}
\end{table}

\section{Membuat Diagram dengan TikZ}

Paket \texttt{TikZ} digunakan untuk membuat gambar vektor langsung di dalam dokumen \LaTeX, seperti diagram blok, flowchart, dan grafik.

\subsection{Diagram Alur (Flowchart)}

Berikut contoh flowchart sederhana:

\begin{figure}[h]
  \centering
  \begin{tikzpicture}[
    node distance=2cm,
    >=Stealth,
    box/.style={draw, rectangle, rounded corners, minimum width=3cm, minimum height=1cm, align=center}
  ]
    \node[box] (start) {Mulai};
    \node[box, below of=start] (process) {Proses};
    \node[box, below of=process] (end) {Selesai};

    \draw[->] (start) -- (process);
    \draw[->] (process) -- (end);
  \end{tikzpicture};
  \caption{Contoh diagram alur sederhana menggunakan TikZ.}
  \label{fig:tikz-flowchart}
\end{figure}

\subsection{Grafik Fungsi}

TikZ juga dapat menggambar grafik fungsi matematika:

\begin{figure}[h]
  \centering
  \begin{tikzpicture}[scale=0.9]
    \draw[->] (-0.5,0) -- (4.5,0) node[below] {$x$};
    \draw[->] (0,-0.5) -- (0,4.5) node[left] {$y$};
    \draw[help lines, step=0.5cm, lightgray] (0,0) grid (4,4);

    \foreach \x in {1,2,3,4}
      \draw (\x,0) -- (\x,-0.1) node[below] {\small \x};
    \foreach \y in {1,2,3,4}
      \draw (0,\y) -- (-0.1,\y) node[left] {\small \y};

    \draw[thick, blue, domain=0:4, samples=50]
      plot (\x, {0.25*\x*\x + 0.5*\x + 0.5})
      node[right] {\small $y = 0{,}25x^{2} + 0{,}5x + 0{,}5$};
  \end{tikzpicture}
  \caption{Contoh grafik fungsi kuadrat menggunakan TikZ.}
  \label{fig:tikz-grafik}
\end{figure}


\section{Menulis Pseudocode dan Algoritma}

Penulisan pseudocode atau algoritma dalam dokumen ilmiah berguna untuk menjelaskan langkah-langkah prosedur secara sistematis. Template ini menggunakan paket \texttt{algorithm2e} yang sudah dikonfigurasi di \verb|main.tex| dengan gaya penomoran per bab (misalnya Algoritma 2.1, 2.2, dst).

\subsection{Struktur Dasar Algoritma}

Berikut adalah struktur dasar penulisan algoritma menggunakan lingkungan \verb|algorithm|:

\begin{verbatim}
\begin{algorithm}[H]
  \caption{Nama Algoritma}
  \label{alg:label-algoritma}
  \KwInput{Deskripsi input}
  \KwOutput{Deskripsi output}

  langkah 1\;
  langkah 2\;
  \Return{hasil}\;
\end{algorithm}
\end{verbatim}

Perintah-perintah penting yang tersedia antara lain:
\begin{itemize}
  \item \verb|\KwInput{...}| -- untuk mendeskripsikan input algoritma.
  \item \verb|\KwOutput{...}| -- untuk mendeskripsikan output algoritma.
  \item \verb|\If{kondisi}{...}| -- untuk struktur kondisional \emph{if}.
  \item \verb|\eIf{kondisi}{...}{p...}| -- untuk struktur \emph{if-else}.
  \item \verb|\While{kondisi}{...}| -- untuk perulangan \emph{while}.
  \item \verb|\For{iterator}{...}| -- untuk perulangan \emph{for}.
  \item \verb|\ForEach{item}{...}| -- untuk perulangan \emph{foreach}.
  \item \verb|\Return{...}| -- untuk mengembalikan nilai.
  \item \verb|\Comment{...}| -- untuk menambahkan komentar.
\end{itemize}

\subsection{Contoh Algoritma Sederhana}

Berikut adalah contoh algoritma pencarian linear sederhana:

\begin{algorithm}[H]
  \caption{Pencarian Linear}
  \label{alg:linear-search}
  \KwInput{Array $A$ dengan $n$ elemen, nilai target $x$}
  \KwOutput{Indeks elemen jika ditemukan, $-1$ jika tidak}

  \For{$i \leftarrow 0$ \KwTo $n-1$}{
    \If{$A[i] = x$}{
      \Return{$i$}\;
    }
  }
  \Return{$-1$}\;
\end{algorithm}

Algoritma \ref{alg:linear-search} menunjukkan proses pencarian linear yang memeriksa setiap elemen array secara berurutan hingga menemukan nilai target atau mencapai akhir array.

\subsection{Contoh Algoritma dengan Struktur Kompleks}

Berikut adalah contoh algoritma yang lebih kompleks dengan penggunaan \emph{if-else} dan komentar:

\begin{algorithm}[H]
  \caption{Binary Search}
  \label{alg:binary-search}
  \KwInput{Array terurut $A$ dengan $n$ elemen, nilai target $x$}
  \KwOutput{Indeks elemen jika ditemukan, $-1$ jika tidak}

  $low \leftarrow 0$\;
  $high \leftarrow n - 1$\;

  \While{$low \leq high$}{
    $mid \leftarrow \lfloor (low + high) / 2 \rfloor$\; \Comment{Hitung indeks tengah}

    \eIf{$A[mid] = x$}{
      \Return{$mid$}\; \Comment{Elemen ditemukan}
    }{
      \eIf{$A[mid] < x$}{
        $low \leftarrow mid + 1$\; \Comment{Cari di bagian kanan}
      }{
        $high \leftarrow mid - 1$\; \Comment{Cari di bagian kiri}
      }
    }
  }

  \Return{$-1$}\; \Comment{Elemen tidak ditemukan}

\end{algorithm}

Algoritma \ref{alg:binary-search} menerapkan teknik \emph{divide and conquer} dengan kompleksitas waktu $O(\log n)$, yang jauh lebih efisien dibandingkan pencarian linear pada array berukuran besar.

\subsection{Merujuk Algoritma dalam Teks}

Untuk merujuk algoritma di dalam teks, gunakan perintah \verb|\ref{alg:label}| atau \verb|\cref{alg:label}|. Contoh:
\begin{itemize}
  \item \verb|Algoritma \ref{alg:binary-search}| menghasilkan: Algoritma \ref{alg:binary-search}
  \item \verb|\cref{alg:binary-search}| menghasilkan: \cref{alg:binary-search}
\end{itemize}

Dengan menggunakan paket \texttt{algorithm2e}, penulis dapat menyajikan algoritma secara profesional dan konsisten dengan format dokumen ilmiah.

\section{Menulis Kode Sumber dengan Syntax Highlighting}

Untuk menampilkan kode sumber dengan pewarnaan sintaks (\emph{syntax highlighting}), template ini menggunakan paket \texttt{listings} yang sudah dikonfigurasi di \verb|main.tex|. Paket ini mendukung berbagai bahasa pemrograman seperti Python, JavaScript, Java, C++, dan banyak lagi.

\subsection{Struktur Dasar Penulisan Kode}

Berikut adalah struktur dasar untuk menampilkan kode dengan caption dan label:

\begin{verbatim}
\begin{lstlisting}[style=pythonstyle, caption={Deskripsi Kode}, label={lst:label-kode}]
# kode program di sini
print("Hello, World!")
\end{lstlisting}
\end{verbatim}

Style yang tersedia antara lain:
\begin{itemize}
  \item \verb|pythonstyle| -- untuk bahasa Python
  \item \verb|javascriptstyle| -- untuk bahasa JavaScript
  \item \verb|javastyle| -- untuk bahasa Java
  \item \verb|defaultstyle| -- style umum untuk bahasa lainnya
\end{itemize}

\subsection{Contoh Kode Python}

Berikut adalah contoh kode Python dengan syntax highlighting:

\begin{lstlisting}[style=pythonstyle, caption={Contoh Fungsi Python untuk Menghitung Faktorial}, label={lst:python-factorial}]
def factorial(n):
    """
    Menghitung faktorial dari bilangan bulat non-negatif.

    Args:
        n: Bilangan bulat non-negatif
    Returns:
        Hasil faktorial dari n
    """
    if n < 0:
        raise ValueError("Input harus bilangan non-negatif")
    elif n == 0 or n == 1:
        return 1
    else:
        result = 1
        for i in range(2, n + 1):
            result *= i
        return result

# Contoh penggunaan
print(f"5! = {factorial(5)}")  # Output: 5! = 120
\end{lstlisting}

Kode \ref{lst:python-factorial} menunjukkan implementasi fungsi faktorial dalam Python dengan dokumentasi lengkap dan penanganan kesalahan.

\subsection{Contoh Kode JavaScript}

Berikut adalah contoh kode JavaScript:

\begin{lstlisting}[style=javascriptstyle, caption={Contoh Fungsi Async JavaScript}, label={lst:js-async}]
// Fungsi async untuk mengambil data dari API
async function fetchUserData(userId) {
    try {
        const response = await fetch(`/api/users/${userId}`);

        if (!response.ok) {
            throw new Error('User not found');
        }

        const userData = await response.json();
        return userData;
    } catch (error) {
        console.error('Error fetching user:', error.message);
        return null;
    }
}

// Contoh penggunaan
const user = await fetchUserData(123);
console.log(user.name);
\end{lstlisting}

\subsection{Contoh Kode Java}

Berikut adalah contoh kode Java:

\begin{lstlisting}[style=javastyle, caption={Contoh Class Java Sederhana}, label={lst:java-class}]
public class Calculator {
    private int result;

    public Calculator() {
        this.result = 0;
    }

    public int add(int a, int b) {
        result = a + b;
        return result;
    }

    public int subtract(int a, int b) {
        result = a - b;
        return result;
    }

    public static void main(String[] args) {
        Calculator calc = new Calculator();
        System.out.println("5 + 3 = " + calc.add(5, 3));
        System.out.println("5 - 3 = " + calc.subtract(5, 3));
    }
}
\end{lstlisting}

\subsection{Kode Inline dalam Teks}

Untuk menampilkan kode pendek di dalam paragraf, gunakan perintah \verb|\lstinline|. Contoh:

\begin{itemize}
  \item \verb|\lstinline{print("Hello")}| menghasilkan: \lstinline{print("Hello")}
  \item \verb|\lstinline[style=pythonstyle]{def main():}| menghasilkan: \lstinline[style=pythonstyle]{def main():}
\end{itemize}

\subsection{Merujuk Kode dalam Teks}

Untuk merujuk kode di dalam teks, gunakan perintah \verb|\ref{lst:label}|. Contoh:
\begin{itemize}
  \item \verb|Kode \ref{lst:python-factorial}| menghasilkan: Kode \ref{lst:python-factorial}
  \item \verb|Kode \ref{lst:js-async}| menghasilkan: Kode \ref{lst:js-async}
\end{itemize}

\subsection{Kustomisasi Tambahan}

Anda juga dapat mengkustomisasi tampilan kode secara langsung dengan opsi tambahan:

\begin{verbatim}
\begin{lstlisting}[
    language=Python,
    numbers=none,          % tanpa nomor baris
    frame=none,            % tanpa border
    backgroundcolor=\color{white}  % latar putih
]
# kode tanpa nomor baris dan border
x = 10
\end{lstlisting}
\end{verbatim}

Dengan paket \texttt{listings}, penulis dapat menyajikan kode sumber secara profesional dengan pewarnaan sintaks yang memudahkan pembaca memahami struktur dan logika program.
      % BAB II  - Tinjauan Pustaka
% !TEX TS-program = xelatex
\chapter{TIPS DAN TRIK}

Bab ini berisi kumpulan tips dan trik praktis untuk menulis dokumen \LaTeX\ dengan lebih efisien. Materi disusun berdasarkan pengalaman umum dan kesalahan yang sering ditemui oleh penulis pemula maupun menengah.

\section{Praktik Terbaik Penulisan}

\subsection{Organisasi File dan Folder}

Menjaga struktur proyek yang rapi sangat penting untuk dokumen besar seperti tesis. Berikut beberapa rekomendasi:

\begin{enumerate}
  \item \textbf{Pisahkan bab ke file terpisah.} Gunakan perintah \verb|\input{chapters/bab1}| untuk memasukkan file eksternal. Ini memudahkan navigasi dan mengurangi risiko konflik saat mengedit.

  \item \textbf{Gunakan folder khusus untuk aset.} Simpan gambar di folder \texttt{images/}, data di \texttt{data/}, dan lampiran di \texttt{backmatter/}. Struktur yang konsisten memudahkan pencarian file.

  \item \textbf{Beri nama file yang deskriptif.} Hindari nama seperti \texttt{gambar1.png}. Gunakan \texttt{arsitektur\_sistem.png} atau \texttt{hasil\_eksperimen\_akurasi.png}.

  \item \textbf{Backup secara berkala.} Gunakan version control seperti Git untuk melacak perubahan. Commit setiap kali menyelesaikan satu bagian penting.
\end{enumerate}

\subsection{Penulisan Konten yang Efektif}

\begin{enumerate}
  \item \textbf{Tulis dulu, format kemudian.} Fokus pada isi terlebih dahulu tanpa terlalu memikirkan tampilan. Formatting dapat dilakukan setelah konten selesai.

  \item \textbf{Gunakan komentar untuk catatan.} Tambahkan \verb|% TODO: ...| atau \verb|% FIXME: ...| untuk menandai bagian yang perlu diperbaiki. Ini memudahkan review.

  \item \textbf{Pisahkan paragraf dengan baris kosong.} Dalam \LaTeX, satu baris kosong menandakan paragraf baru. Hindari menggunakan \verb|\\| untuk membuat paragraf.

  \item \textbf{Hindari hard-coded formatting.} Jangan gunakan \verb|\vspace{2cm}| secara berlebihan. Biarkan \LaTeX\ mengatur jarak secara otomatis.
\end{enumerate}

\section{Mengatasi Masalah Umum}

\subsection{Error Kompilasi}

Berikut beberapa error umum dan cara mengatasinya:

\begin{enumerate}
  \item \textbf{Undefined control sequence.} Error ini muncul ketika perintah tidak dikenali. Pastikan paket yang diperlukan sudah dimuat dengan \verb|\usepackage{...}|.

  \item \textbf{Missing \$ inserted.} Terjadi ketika simbol matematika ditulis di luar mode math. Gunakan \verb|$...$| untuk inline math atau \verb|\[...\]| untuk display math.

  \item \textbf{Misplaced alignment tab character \&.} Karakter \verb|&| hanya boleh digunakan dalam lingkungan tabel atau array. Di luar itu, gunakan \verb|\&|.

  \item \textbf{File not found.} Periksa path file dan pastikan tidak ada typo. Path bersifat case-sensitive di Linux/macOS.

  \item \textbf{Missing number, treated as zero.} Biasanya terjadi pada perintah yang memerlukan angka, seperti \verb|\includegraphics[width=]|. Pastikan nilai diberikan dengan lengkap.
\end{enumerate}

\subsection{Masalah Referensi dan Sitasi}

\begin{enumerate}
  \item \textbf{Referensi muncul sebagai [?].} Ini berarti label belum terdefinisi. Kompilasi dokumen dua kali atau pastikan \verb|\label{...}| sudah ada.

  \item \textbf{Sitasi tidak muncul di daftar pustaka.} Jalankan Biber setelah kompilasi pertama, lalu kompilasi lagi dengan \LaTeX. Urutan: XeLaTeX $\rightarrow$ Biber $\rightarrow$ XeLaTeX $\rightarrow$ XeLaTeX.

  \item \textbf{Key tidak ditemukan dalam .bib file.} Periksa apakah key yang digunakan di \verb|\cite{...}| sama persis dengan yang ada di file \texttt{references.bib}.
\end{enumerate}

\section{Tips Produktivitas}

\subsection{Shortcut dan Snippet}

\begin{enumerate}
  \item \textbf{Buat snippet untuk struktur berulang.} Jika sering menulis lingkungan \verb|figure|, buat shortcut di editor Anda untuk menyisipkan template lengkap.

  \item \textbf{Gunakan auto-completion.} Editor seperti VS Code dengan ekstensi LaTeX Workshop menyediakan auto-complete untuk perintah \LaTeX.

  \item \textbf{Manfaatkan find-and-replace dengan regex.} Untuk perubahan massal, regex sangat membantu. Contoh: mengubah semua \verb|\textbf{...}| menjadi \verb|\emph{...}|.
\end{enumerate}

\subsection{Mempercepat Kompilasi}

\begin{enumerate}
  \item \textbf{Gunakan mode draft.} Tambahkan opsi \verb|draft| pada \verb|\documentclass| untuk melewati loading gambar saat proofreading.

  \item \textbf{Kompilasi bagian tertentu saja.} Gunakan \verb|\includeonly{chapters/bab1}| untuk hanya mengompilasi bab tertentu. Ini menghemat waktu untuk dokumen besar.

  \item \textbf{Nonaktifkan paket yang tidak diperlukan.} Komentar paket yang belum digunakan dengan \verb|%| untuk mempercepat kompilasi.
\end{enumerate}

\section{Optimasi Tampilan}

\subsection{Mengatasi Overfull dan Underfull Box}

\begin{enumerate}
  \item \textbf{Overfull hbox.} Kata terlalu panjang dan melampaui margin. Solusi:
  \begin{itemize}
    \item Gunakan \verb|\hyphenation{kata-pan-jang}| untuk mengatur pemenggalan.
    \item Tambahkan \verb|\sloppy| di awal paragraf bermasalah.
    \item Reword kalimat agar lebih pendek.
  \end{itemize}

  \item \textbf{Underfull hbox.} Spasi antar kata terlalu lebar. Solusi:
  \begin{itemize}
    \item Tambahkan teks atau reword kalimat.
    \item Gunakan \verb|\mbox{...}| untuk mencegah pemisahan kata.
  \end{itemize}
\end{enumerate}

\subsection{Penempatan Float yang Optimal}

\begin{enumerate}
  \item \textbf{Biarkan \LaTeX\ memilih posisi.} Gunakan \verb|[htbp]| daripada memaksa dengan \verb|[H]|. \LaTeX\ biasanya membuat keputusan layout yang baik.

  \item \textbf{Tempatkan float setelah referensi pertama.} Ini membantu pembaca menemukan gambar/tabel dengan mudah.

  \item \textbf{Gunakan \texttt{placeins} untuk membatasi float.} Paket ini mencegah float melayang terlalu jauh dari posisi aslinya.
\end{enumerate}

\section{Keamanan dan Backup}

\subsection{Menghindari Kehilangan Data}

\begin{enumerate}
  \item \textbf{Simpan di cloud.} Gunakan Google Drive, Dropbox, atau OneDrive untuk sinkronisasi otomatis.

  \item \textbf{Gunakan Git.} Version control memungkinkan Anda kembali ke versi sebelumnya jika terjadi kesalahan.

  \item \textbf{Buat backup file .bib terpisah.} File bibliografi adalah aset penting yang sulit dibuat ulang.

  \item \textbf{Ekspor PDF secara berkala.} Simpan salinan PDF untuk berjaga-jaga jika file sumber rusak.
\end{enumerate}

\section{Referensi Cepat Perintah Penting}

Berikut ringkasan perintah yang sering digunakan:

\begin{table}[h]
  \centering
  \caption{Referensi Cepat Perintah \LaTeX}
  \label{tab:referensi-cepat}
  \begin{tabular}{ll}
    \toprule
    \textbf{Fungsi} & \textbf{Perintah} \\
    \midrule
    Teks tebal & \verb|\textbf{teks}| \\
    Teks miring & \verb|\textit{teks}| \\
    Inline math & \verb|$x^2 + y^2$| \\
    Display math & \verb|\[ E = mc^2 \]| \\
    Sitasi & \verb|\cite{key}| \\
    Referensi & \verb|\ref{label}| \\
    Referensi persamaan & \verb|\eqref{eq:label}| \\
    Non-breaking space & \verb|Gambar~\ref{fig:x}| \\
    Komentar & \verb|% ini komentar| \\
    Garis baru (dalam tabel) & \verb|\\| \\
    \bottomrule
  \end{tabular}
\end{table}

Dengan menerapkan tips dan trik di atas, proses penulisan dokumen \LaTeX\ akan menjadi lebih efisien dan menyenangkan. Ingatlah bahwa keahlian \LaTeX\ berkembang seiring praktik—semakin banyak menulis, semakin terbiasa dengan perintah dan konvensinya.
     % BAB III - Metode Penelitian
%% chtex-lint: disable
\chapter{PELAKSANAAN PENELITIAN DAN HASIL}
    % BAB IV  - Pelaksanaan dan Hasil
%% chtex-lint: disable
\chapter{PENUTUP}
              % BAB V   - Penutup

% ############################################################################
%                         BAGIAN III: BACKMATTER
%                       (Daftar Pustaka & Lampiran)
% ############################################################################

% ============================================================================
% DAFTAR PUSTAKA
% ============================================================================
% PETUNJUK: Edit file references.bib untuk menambah/mengubah referensi
%           Gunakan format BibLaTeX dengan style IEEE
% ============================================================================
\clearpage
\phantomsection
\printbibliography[heading=bibintoc, title={DAFTAR PUSTAKA}]

% ============================================================================
% LAMPIRAN
% ============================================================================
% PETUNJUK: Edit file-file berikut di folder backmatter/
%   - lampiran_a.tex : Lampiran A
%   - lampiran_b.tex : Lampiran B (tambahkan sesuai kebutuhan)
%   - lampiran_c.tex : Lampiran C (tambahkan sesuai kebutuhan)
%
% Uncomment (hapus %) pada baris yang sudah selesai ditulis
% ============================================================================
\appendix
% chtex-lint: disable
\chapter*{LAMPIRAN}
\addcontentsline{toc}{chapter}{LAMPIRAN}
\section*{Lampiran: Laporan Turnitin}
\myappendix{1}{Laporan Turnitin}

\noindent
\includegraphics[width=\textwidth, page=1]{documents/opsmanual.pdf}


               % Lampiran A
%\input{backmatter/lampiran_b}              % Lampiran B
%\input{backmatter/lampiran_c}              % Lampiran C

\end{document}
